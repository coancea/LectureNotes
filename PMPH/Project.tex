\documentclass{beamer}

%% \documentclass[handout]{beamer}
%% % use this with the [handout] option to create handouts for the audience
%% \usepackage{pgfpages}
%% \pgfpagesuselayout{2 on 1}[a4paper,border shrink=5mm]

\mode<presentation>
{
  \usetheme{Diku}
% set this to your preferences:
  \setbeamercovered{invisible}
%  \setbeamercovered{transparent}
}

\usepackage{graphicx}
\usepackage{epic}

\usepackage{amsmath}
\usepackage{amssymb}
\usepackage{amsthm}

\newcommand{\basetop}[1]{\vtop{\vskip-1ex\hbox{#1}}}
\newcommand{\source}[1]{\let\thefootnote\relax\footnotetext{\scriptsize\textcolor{kugray1}{Source: #1}}}

% for coloured code citation in text:
\usepackage{fancyvrb}

%%%%%%%%%%%%%%%%%%%%%%%%%%%%%%%%%
%%%%%    code sections   %%%%%%%%
%%%%%%%%%%%%%%%%%%%%%%%%%%%%%%%%%

% code highlighting commands in own block
\DefineVerbatimEnvironment{code}{Verbatim}{fontsize=\scriptsize}
\DefineVerbatimEnvironment{icode}{Verbatim}{fontsize=\scriptsize}

% Fancy code with color commands:
\DefineVerbatimEnvironment{colorcode}%
        {Verbatim}{fontsize=\scriptsize,commandchars=\\\{\}}

%%%%%%%%%%%%%%%%%%%%%%%%%%%%%%%%%%
%%%%%    some coloring    %%%%%%%%

\definecolor{Red}{RGB}{220,50,10}
\definecolor{Blue}{RGB}{0,51,102}
\definecolor{Yellow}{RGB}{102,51,0}
\definecolor{Orange}{RGB}{178,36,36}
\definecolor{Grey}{RGB}{180,180,180}
\definecolor{Green}{RGB}{20,120,20}
\definecolor{Purple}{RGB}{160,50,100}
\newcommand{\red}[1]{\textcolor{Red}{{#1}}}
\newcommand{\blue}[1]{\textcolor{Blue}{{#1}}}
\newcommand{\yellow}[1]{\textcolor{Yellow}{{#1}}}
\newcommand{\orange}[1]{\textcolor{Orange}{{#1}}}
\newcommand{\grey}[1]{\textcolor{Grey}{{#1}}}
\newcommand{\green}[1]{\textcolor{Green}{{#1}}}
\newcommand{\purple}[1]{\textcolor{Purple}{{#1}}}




% use "DIKU green" from our color theme for \emph
\renewcommand{\emph}[1]{\textcolor{structure}{#1}}
% use some not-too-bright red for an \emp command
\definecolor{DikuRed}{RGB}{130,50,32}
\newcommand{\emp}[1]{\textcolor{DikuRed}{ #1}}
\definecolor{CosGreen}{RGB}{10,100,70}
\newcommand{\emphh}[1]{\textcolor{CosGreen}{ #1}}
\definecolor{CosBlue}{RGB}{55,111,122}
\newcommand{\emphb}[1]{\textcolor{CosBlue}{ #1}}
\definecolor{CosRed}{RGB}{253,1,1}
\newcommand{\empr}[1]{\textcolor{CosRed}{ #1}}

\newcommand{\mymath}[1]{$ #1 $}
\newcommand{\myindx}[1]{_{#1}}
\newcommand{\myindu}[1]{^{#1}}

\newcommand{\Fasto}{\textsc{Fasto}\xspace}


%%%%%%%%%%%%%%%%%%%%

\title[Project]{Project Related Discussion}

\author[C.~Oancea]{Cosmin E. Oancea\\{\tt cosmin.oancea@diku.dk}}

\institute{Department of Computer Science (DIKU)\\University of Copenhagen}


\date[Sept 2014]{September 2014 PMPH Lecture Notes}


\begin{document}

\titleslide

%%%%%%%% real content starts here %%%%%%%%%%

\section{Code Structure}

\begin{frame}[fragile]
	\tableofcontents[currentsection]
\end{frame}


\begin{frame}[fragile,t]
  \frametitle{Datasets} % of CPU, Multicores, GPGPU
\begin{itemize}
    \item[Small:] OUTER=16, NUM\_X=32, NUM\_Y=256, NUM\_T=90
    \item{Medium:} OUTER=32, NUM\_X=47, NUM\_Y=181, NUM\_T=93
    \item{Large:} OUTER=128, NUM\_X=256, NUM\_Y=256, NUM\_T=128
\end{itemize}

\end{frame}


\begin{frame}[fragile,t]
  \frametitle{Code Structure} % of CPU, Multicores, GPGPU

\begin{block}{Code Entry Point}
\begin{columns}
\column{0.48\textwidth}
\begin{colorcode}
void   run_OrigCPU(...) \{
  REAL strike;
  PrivGlobs globs(numX,numY,numT);
  \emph{for(int i=0; i<outer; ++ i)} \{
    strike = 0.001*i;
    res[i] = \emp{value}( globs,s0,strike,t,
                    alpha,nu,   beta,
                    numX, numY, numT );
  \}
\}
\end{colorcode}
\column{0.48\textwidth}
\begin{colorcode}
REAL   \emp{value}( ... ) \{
  initGrid(s0,alpha,nu,t, 
           numX,numY,numT, 
           globs);
  initOperator(globs.myX,
               globs.myDxx);
  initOperator(globs.myY,
               globs.myDyy);
  setPayoff(strike, globs);
  \alert{for(int i=numT-2;i>=0;--i)}\{
    updateParams(i,alpha,beta,
                 nu,globs);
    rollback(i, globs);
  \}
  return globs.myResult[globs.myXindex]
                       [globs.myYindex];
\}
\end{colorcode}
\end{columns}
\end{block} 

\end{frame}


\begin{frame}[fragile,t]
  \frametitle{Loop Nests} % of CPU, Multicores, GPGPU

\begin{block}{Loop Nests}
\begin{columns}
\column{0.95\textwidth}
\begin{colorcode}
rollback( ... ) \{
  vector<vector<REAL> > u(numY, vector<REAL>(numX));   // [numY][numX]
  vector<vector<REAL> > v(numX, vector<REAL>(numY));   // [numX][numY]
  vector<REAL> a(numZ), b(numZ), c(numZ), y(numZ);     // [max(numX,numY)] 
  vector<REAL> yy(numZ);  // temporary used in tridag  // [max(numX,numY)]
  for(i=0;i<numX;i++) \{
    for(j=0;j<numY;j++) \{
      u[j][i] = dtInv*\blue{globs.\alert{myResult[i][j]}};
  \} \}  .......
  // implicit y
  for(i=0;i<numX;i++) \{ 
    for(j=0;j<numY;j++) \{  // here a, b, c should have size [numY]
      a[j] =		 - 0.5*(0.5*globs.myVarY[i][j]*globs.myDyy[j][0]);
      b[j] = dtInv - 0.5*(0.5*globs.myVarY[i][j]*globs.myDyy[j][1]);
      c[j] =		 - 0.5*(0.5*globs.myVarY[i][j]*globs.myDyy[j][2]);
    \}
    for(j=0;j<numY;j++)
      y[j] = dtInv*u[j][i] - 0.5*v[i][j];
    // here yy should have size [numY]
    tridag(a,b,c,y,numY,globs.\alert{myResult[i]},yy);
  \} \}
\end{colorcode}
\column{0.05\textwidth}
\end{columns}
\end{block} 

\end{frame}


\begin{frame}[fragile,t]
  \frametitle{How To Parallelize} % of CPU, Multicores, GPGPU

\begin{itemize}
    \item summarize accesses inter-procedurally. 
            For each loop what does it write and what does it read?\medskip
    \item Within each loop: are all reads covered by writes
            happening within the same iteration?
            If so then privatization solves those dependencies!\medskip
    \item Decide for each loop whether it can or cannot be parallelize.\medskip
    \item Use loop distribution to create perfect nests,
            which will become later your CUDA kernels.\medskip
    \item Use loop interchange and/or matrix transposition
            to obtain coalesced access to global memory.
\end{itemize}
\end{frame}

\end{document}

