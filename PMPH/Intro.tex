\documentclass{beamer}

%% \documentclass[handout]{beamer}
%% % use this with the [handout] option to create handouts for the audience
%% \usepackage{pgfpages}
%% \pgfpagesuselayout{2 on 1}[a4paper,border shrink=5mm]

\mode<presentation>
{
  \usetheme{Diku}
% set this to your preferences:
  \setbeamercovered{invisible}
%  \setbeamercovered{transparent}
}

\usepackage{graphicx}
\usepackage{epic}

\usepackage{amsmath}
\usepackage{amssymb}
\usepackage{amsthm}

\newcommand{\basetop}[1]{\vtop{\vskip-1ex\hbox{#1}}}
\newcommand{\source}[1]{\let\thefootnote\relax\footnotetext{\scriptsize\textcolor{kugray1}{Source: #1}}}

% for coloured code citation in text:
\usepackage{fancyvrb}

%%%%%%%%%%%%%%%%%%%%%%%%%%%%%%%%%
%%%%%    code sections   %%%%%%%%
%%%%%%%%%%%%%%%%%%%%%%%%%%%%%%%%%

% code highlighting commands in own block
\DefineVerbatimEnvironment{code}{Verbatim}{fontsize=\scriptsize}
\DefineVerbatimEnvironment{icode}{Verbatim}{fontsize=\scriptsize}

% Fancy code with color commands:
\DefineVerbatimEnvironment{colorcode}%
        {Verbatim}{fontsize=\scriptsize,commandchars=\\\{\}}

%%%%%%%%%%%%%%%%%%%%%%%%%%%%%%%%%%
%%%%%    some coloring    %%%%%%%%

\definecolor{Red}{RGB}{220,50,10}
\definecolor{Blue}{RGB}{0,51,102}
\definecolor{Yellow}{RGB}{102,51,0}
\definecolor{Orange}{RGB}{178,36,36}
\definecolor{Grey}{RGB}{180,180,180}
\definecolor{Green}{RGB}{20,120,20}
\definecolor{Purple}{RGB}{160,50,100}
\newcommand{\red}[1]{\textcolor{Red}{{#1}}}
\newcommand{\blue}[1]{\textcolor{Blue}{{#1}}}
\newcommand{\yellow}[1]{\textcolor{Yellow}{{#1}}}
\newcommand{\orange}[1]{\textcolor{Orange}{{#1}}}
\newcommand{\grey}[1]{\textcolor{Grey}{{#1}}}
\newcommand{\green}[1]{\textcolor{Green}{{#1}}}
\newcommand{\purple}[1]{\textcolor{Purple}{{#1}}}




% use "DIKU green" from our color theme for \emph
\renewcommand{\emph}[1]{\textcolor{structure}{#1}}
% use some not-too-bright red for an \emp command
\definecolor{DikuRed}{RGB}{130,50,32}
\newcommand{\emp}[1]{\textcolor{DikuRed}{ #1}}
\definecolor{CosGreen}{RGB}{10,100,70}
\newcommand{\emphh}[1]{\textcolor{CosGreen}{ #1}}
\definecolor{CosBlue}{RGB}{55,111,122}
\newcommand{\emphb}[1]{\textcolor{CosBlue}{ #1}}
\definecolor{CosRed}{RGB}{253,1,1}
\newcommand{\empr}[1]{\textcolor{CosRed}{ #1}}

\newcommand{\mymath}[1]{$ #1 $}
\newcommand{\myindx}[1]{_{#1}}
\newcommand{\myindu}[1]{^{#1}}

\newcommand{\Fasto}{\textsc{Fasto}\xspace}


%%%%%%%%%%%%%%%%%%%%

\title[Intro]{Introduction: Hardware Trends\\and Vector Machines}

\author[C.~Oancea]{Cosmin E. Oancea {\tt cosmin.oancea@diku.dk}}

\institute{Department of Computer Science (DIKU)\\University of Copenhagen}


\date[Sept 2014]{September 2014 PMPH Lecture Notes}


\begin{document}

\titleslide


%%%%%%%%%%%%%%%%%%%%%%%%%%%%%%%%%%%%%%%%%%%%%%%%%%%%%%%%%%%%%%%%%%%%%%
%%%%%%%%%%%%%%%%%%%%%%%%%%%%%%%%%%%%%%%%%%%%%%%%%%%%%%%%%%%%%%%%%%%%%%
%%%%%%%%%%%%%%%%%%%%%%%%%%%%%%%%%%%%%%%%%%%%%%%%%%%%%%%%%%%%%%%%%%%%%%
\begin{frame}[fragile]
	\tableofcontents
\end{frame}

%%%%%%%%%%%%%%%%%%%%%%%%%%%%%%%%%%%%%%%%%%%%%%%%%%%
%%%%%%%%%%%%%%%%%%%%%%%%%%%%%%%%%%%%%%%%%%%%%%%%%%%
%%%%%%%%%%%%%%%%%%%%%%%%%%%%%%%%%%%%%%%%%%%%%%%%%%%

%\section{Scalable Shared Memory Systems}

\section{Introduction$^{\mbox{\bf 2}}$}

\subsection{Brief History}

\begin{frame}[fragile,t]
\frametitle{Introduction$^{\mbox{\bf 2}}$}

\begin{itemize}
    \item Past $>$ 20 years \emph{Information Revolution}:\\
        Explosive growth of semiconductor integration + Internet.\bigskip

    \item Moore's Low 1960s: 
    \begin{itemize}
        \item computing power doubles every 19-24 months 
        \item \emph{system cost effectiveness = {\tt performance/cost} 
                grows exp}.
        \item \alert{CMOS endpoint is near}: miniaturization reaches its limits.   
    \end  {itemize}\bigskip

    \item Improved Chip Design:
    \begin{itemize}
        \item each new process generation $\Rightarrow$ higher clock rates 
        \item logic switching speed \& amount of on-chip logic increase ($\uparrow$)
        \item better circuit design \& deep pipelines $\Rightarrow$ 
                fewer gate delays / stage
        \item $\uparrow$ on-chip resources $\Rightarrow$ $\uparrow$ throughput by 
                parallelism at all stages. 
    \end  {itemize}\bigskip

    \item \emp{Computer Arch 1970s: How to Best Use the ever-increasing ($\uparrow$) 
            Wealth of Resources?}

\end{itemize}

\end{frame}

\begin{frame}[fragile,t]
\frametitle{Brief History}

\begin{itemize}
        \item \emph{ICPP, ISCA 1980/90s: parallel architectures popular topic.}\\
              Demise of SingleCPU System: inevitable \& fast approaching.\bigskip

        \item \alert{Whatever happened? Mid90 Killer-Micro:}
        \begin{itemize}
            \item The rapid increase ($\uparrow$) transistor density $\Rightarrow$\\
            \item path of least resistance: ever-increasing speed of SingleCPU
            \item Complex Out-Of-Order (OoO) processors: 100s instructions/cycle.
            \item Commercial arena: multiprocessors just an uniprocessor extension.
        \end  {itemize}\bigskip

        \item \alert{What Changed?} Multiprocessors Trend: Academia \& Industry:
        \begin{itemize}
            \item \emp{power complexity}
            \item \emp{Memory WALL}: $\uparrow$ performance gap between processor \& memory 
        \end  {itemize}\bigskip

       \item \emph{All Future Architectures adopt some form of massive parallelism!}
\end{itemize}

\end{frame}


\subsection{Computer Architecture Definition}

\begin{frame}[fragile,t]
\frametitle{What used to Be Computer Architecture?}

\begin{columns}\hspace{-8ex}
\column{0.6\textwidth}
\includegraphics[width=44ex]{Ch1Figs/SysOrg}
\column{0.65\textwidth}\vspace{-3ex}
\begin{itemize}
    \item Multi-Layered: focus competences.
    \item Each layer uses the ones below it.
    \item \emp{Application} C++/Java/SML/Haskell.
    \item \emp{Compiler}: machine code + OpSys calls.
    \item \emp{OpSys}: extends hardware funct \&
              orchestrates resource sharing among multiple users.
    \item \emp{ISA \& Computer Organization}: 
            software-hardware interface. 
\end{itemize}
\end{columns}

\emp{Old Definition} ISA -- critical role in the success of computer industry:
\begin{itemize}
    \item Early on, used to be the hallmark of system design $\Rightarrow$
    \item Non-Portable Software \& No Compiler at that time
    \item 1960s IBM System360 ISA guarantees backward compatibility $\Rightarrow$
    \item Strategy endured test of time \& Behemoth company today
\end{itemize}
\end{frame}



\begin{frame}[fragile,t]
\frametitle{What Is Computer Architecture Today?}

\begin{columns}
\column{0.59\textwidth}
\hspace{-6ex}\includegraphics[width=44ex]{Ch1Figs/SimpleCPU}

\includegraphics[width=39ex]{Ch1Figs/SimpleInterconnect}
\column{0.55\textwidth}\vspace{-17ex}
\begin{itemize}
    \item \emp{Much Broader Definition}: how best to build the computer 
            (includes ISA, but focus changed on organization).\medskip

    \item North Bridge: sys bus connects cores to memory \& IO devs.
    \item PCI bus: IO bus connecting North Bridge to high-speed IO 
            to disk, network \& slower dev\medskip

    \item $\leftarrow$ Generic Multiprocessor with Distributed Memory\medskip

    \item \alert{Parallel Sys Main Components:
                (1) Processor, (2) Memory Hierarchy and 
                (3) Interconnection}
\end{itemize}
\end{columns}

\end{frame}



\begin{frame}[fragile,t]
\frametitle{Software-Hardware Synergy \& Biggest Challenge}
\medskip
\begin{columns}
\column{0.33\textwidth}
\includegraphics[width=29ex]{Ch1Figs/Synergy}
\column{0.63\textwidth}\vspace{-3ex}
\begin{itemize}
    \item Computer Architect Role:
    \item \emp{design trade-offs across HW/SW interf to meet functional 
            \& performance requirements within cost constraints.}
\end{itemize}
\end{columns}
\medskip

\begin{itemize}
    \item software: flexible approach to simplifying hardware, 
            e.g., TLB exceptions, FP ops,
    \item hardware faster $\Rightarrow$ world of tradeoff in between 
    \item guided by the common case $\Rightarrow$ trends are important.\bigskip

    \item \emp{Important Juncture:} 
        \begin{itemize}
            \item \emph{Higher performance requires Parallel Hardware}
            \item \alert{Biggest Challenge: 
                    develop Effective Massively Parallel Software!}
        \end  {itemize}
\end{itemize}
\end{frame}


\subsection{Course Organization}
\begin{frame}
\frametitle{Course Organization}

\begin{tabular}{lccccc}
W  & HARDWARE  & & SOFTWARE     & & LAB/CUDA \\\hline\hline
1 & \alert{Trends}         &                         & \emp{List HOM}     & & \emph{Intro \& Simple}\\
  & \alert{Vector Machine} & \emph{$\longleftarrow$} & \emp{(Map-Reduce)} & & \emph{Map Programming}\\\hline
%
2 & In Order & $\longrightarrow$ & VLIW Instr   & & Scan \&\\
  & Processor& $\longleftarrow$ & Scheduling   & & Reduce \\\hline
%
3 & Cache     & & Reasoning About     & & Sparse Vect\\
  & Coherence & & Parallelism   & & Matrix Mult\\\hline
%
4 & Interconnection & & Case Studies \&   & & Transpose \& Matrix\\
  & Networks        & & Optimizations   & & Matrix Mult\\\hline
%
5 & Memory      & & Optimising   & & Sorting \& Profiling \& \\
  & Consistency & & Locality     & & Mem Optimizations \\\hline
%
6 & OoO, Spec   & & Thread-Level   & & Project \\
  & Processor   & & Speculation    & & Work    \\\hline

%\framebox{Processor}       & & \framebox{Low-Level\\Optimizations}        & & \framebox{CUDA: Scan\\Reduce}\\
%$\downarrow$ && $\uparrow$ \\
%\framebox{\red Intermediate code generation} &$\longrightarrow$ & Intermediate code
\end{tabular}
\medskip
%\alert{Keywords: Reasoning, Tradeoffs, Common Case, }

Three narative threads: the path to complex \& good design: 
\begin{itemize}
    \item \emp{Design Space} tradeoffs, constraints, common case, trends.
    \item \emp{Reasoning}: from simple to complex, \emp{Applying Concepts}.
\end  {itemize}
\end{frame}



%%%%%%%%%%%%%%%%%%%%%%%%%%%%%%%%%%%%%%%%%%%%%

\section{Trends of Critical Components of a Parallel System}

\begin{frame}[fragile]
	\tableofcontents[currentsection]
\end{frame}

\subsection{Processor}

\begin{frame}[fragile,t]
\frametitle{Abstractions}
\medskip

\begin{itemize}
%    \item[Program] set of statements performing computational steps.
%    \item[Process/Thread] embeds the execution of the computation.
    \item A \emp{program} is to a \emp{process/thread} 
            what a recipe is for cooking.\smallskip

    \item \emp{Processor (core)}: hardware entity capable of
            sequencing \& executing thread's instructions.\smallskip

    \item \emp{MT Cores} multiple threads, each 
            running in its thread context.\smallskip

    \item \emp{Multiprocessor:} set of cores connected to execute a workload
        \begin{itemize}
            \item mass produced, off-the-shelf, each several cores \& levels of cache
            \item trend towards migrating system functions on the chip:\\
                    memory controllers, external cache directories, network interface
        \end  {itemize}
\end{itemize}

%\begin{columns}
%\column{0.33\textwidth}
%\includegraphics[width=29ex]{Ch1Figs/Synergy}
%\column{0.63\textwidth}\vspace{-3ex}
%\begin{itemize}
%    \item Computer Architect Role:
%    \item \emp{design trade-offs across HW/SW interf to meet functional 
%            \& performance requirements within cost constraints.}
%\end{itemize}
%\end{columns}

\end{frame}


\begin{frame}[fragile,t]
\frametitle{Processor: Clock Frequency/Rate}

Historically the clock rate (at which instr are executed) has increased 
exponentially (1990-2004).

\begin{columns}
\column{0.65\textwidth}
\hspace{-5ex}\includegraphics[width=52ex]{Ch1Figs/FreqGraph}
\column{0.48\textwidth}\vspace{-3ex}
\begin{scriptsize}
        \begin{itemize}
            \item $1.19\times$ per-year due technology scaling
                    (same hwd on new techn).

            \item 1990-2002: doubled every $21$ months
                    \alert{Curve $1.49\times$: $30$GHz in 2008!}
            
            \item $1.49\times - 1.19\times$: very-deep (10-20 stages) pipelines!
                    ILP via speculative OoO: register renaming, 
                        reorder buffs, branch prediction, 
                        lockup-free caches, memory disambiguation, etc. 

            \item 2004: Intel cancels Pentium4 @4Ghz \&
                    \emph{Switched Track to Multi-Cores} $\Rightarrow$\\
                    \alert{Tectonic Shift away from muscled deeply-pipelined 
                    uniprocessor.}

            \item \emp{Peaked in 2005, but mostly stalled since 2002!}
                     
        \end  {itemize}
\end{scriptsize}
\end{columns}

\end{frame}


\begin{frame}[fragile,t]
\frametitle{Closer Look at Clock Rate}

\begin{columns}
\column{0.5\textwidth}
\includegraphics[width=40ex]{Ch1Figs/FreqGraph}
\column{0.57\textwidth}
        \begin{itemize}
            \item \emph{Technology} (process shrinkage):\smallskip
                \begin{itemize}
                    \item every generation 
                            transistors' \emph{switching speed increases $41\%$}.

                    \item Impact blunted in the future due to \emp{wire delays}
                                (do not scale)
%                            because speed of wire transmission grows much slower than
%                            switching speed.
                \end  {itemize}\medskip
            \item \emph{Pipeline Depth:} more stages $\Rightarrow$\smallskip 
                            less complex $\Rightarrow$ less gates 
                \begin{itemize}
                    \item \# of gate(s delays) dropped by $25\%$ every process generation.
                \end  {itemize}\medskip

                \item \emph{Improved Circuit Design} 
        \end  {itemize}\bigskip
\end{columns}

Clock Rate Increase is Not Sustainable:
\begin{itemize}
    \item \emp{Deeper pipelines}: difficult to build useful stages with $<$ 10 gates
    \item \emp{Wire delays:} wire-transm speed $\uparrow$ much 
            slower than switching,
    \item Circuits clocked at \emp{higher rates consume more power}!  
\end  {itemize}

\end{frame}

\begin{frame}[fragile,t]
\frametitle{Processor: Feature Size \& Number of Transistors}

\begin{columns}
\column{0.65\textwidth}
\includegraphics[width=47ex]{Ch1Figs/FeatureSize}
\column{0.4\textwidth}
%\begin{scriptsize}
        \begin{itemize}
            \item new process generation every 2 years\smallskip
                
            \item feature size reduced $30\%$ every generation

            \item \# of transistors doubles every 2 years (Moore's low).\\
                    1 Billion in 2008. 
        \end  {itemize}
%\end{scriptsize}
\end{columns}
\vspace{-2ex}

Each process generation offers new resources.
\emp{How best to use the $>100$ billion transistors? Large-Scale CMPs (100s-1000s cores)}:
\begin{itemize}
    \item more cache, better memory-system design
    \item fetch and decode multiple instr per clock
    \item running multiple threads per core and on multiple cores
\end  {itemize} 

\end{frame}

\subsection{Memory}

\begin{frame}[fragile,t]
\frametitle{Memory Systems}

\begin{itemize}
    \item \emp{(Main) Memory Wall:} growing gap between processor and memory speed.
            Processor cannot execute faster than memory system can deliver data 
            and instructions!\bigskip

    \item Want Big, Fast \& Chip Memory System\smallskip
    \begin{itemize}
        \item access time 
                increases with size as it is dominated by 
                wire delays$\Rightarrow$ this will not change in future technologies\smallskip
%(address decoding, address line propagation, bit-line propagation) 
        \item multi-level hierarchies (relies on principle of locality)\smallskip
        \item efficient management is KEY, e.g., cache coherence.\smallskip
        \item Cost and Size memories in a basic PC in 2008:
    \end  {itemize} 
\end{itemize}
\bigskip

\begin{tabular}{|l|l|l|l|l|}\hline
Memory & Size  & Marginal Cost & Cost Per MB & Access Time \\\hline
L2 Cache & 1MB & \$20/MB & \$20 & 5nsec \\\hline
Main Memory & 1 GB & \$50/GB & 5c & 200 nsec \\\hline
Disk & 500GB & \$100/500GB & 0.02c & 5 msec \\\hline
\end{tabular}

\end{frame}


\begin{frame}[fragile,t]
\frametitle{Memory Wall? Which Memory Wall??}

\begin{itemize}
            \item \emph{DRAM density increases $4\times$ every 3 years, BUT} \smallskip

            \item \emp{DRAM speed $\uparrow$ only with $7\%$ per year!} 
                    (processor speed by $50\%$) 

            \item \alert{Perception was that Memory Wall will last forever!}

            \item \emph{Memory Wall Stopped Growing around 2002}.
    
            \item Multi/Many-Cores $\Rightarrow$ shifted from Latency 
                    to \emp{Bandwidth WALL}
\end  {itemize}
\vspace{-3ex}

\begin{columns}
\column{0.65\textwidth}
\includegraphics[width=50ex]{Ch1Figs/MemWall}
\column{0.3\textwidth}
\begin{scriptsize}
\begin{itemize}
\item {\tt MemoryWall = mem\_cycle/ proc\_cycle} \smallskip
\item[1990] {\tt MemoryWall = 4} (25MHz,150ns)
\item[2002] exponential growth {\tt MemoryWall = 200} 
\item Stalled since then.
\item If trend continues: 1 Terabit Main Memory by 2021.
\end{itemize}
\end{scriptsize}
\end{columns}

\end{frame}


\begin{frame}[fragile,t]
\frametitle{Disk Memory}

\begin{itemize}
            \item \emph{Historically disk performance improved by $40\%$ per year} \smallskip

            \item {\tt DiskTime=AccessTime+TransferTime} ({\scriptsize {\tt AccessTime=Seek+Latency}})

            \item Historically, transfer time have dominated, but

            \item Today: transfer and access time are of the same \alert{msecs} order
    
            \item Future, Access Time will dominate, but proc-disk gap still large
\end  {itemize}

\begin{columns}
\column{0.5\textwidth}
\includegraphics[width=33ex]{Ch1Figs/DISK}
\column{0.5\textwidth}
\includegraphics[width=33ex]{Ch1Figs/Disk2}
\end{columns}

{\scriptsize Seek Time: head to reach right track, latency: time to reach the first record on track, both depend on rotation speed \& independent on block size}.

\end{frame}

\subsection{Interconnect}
\begin{frame}[fragile,t]
\frametitle{Interconnection Networks}

Present at many layers:
\begin{itemize}
            \item \emp{On-Chip Interconnects:} forward values between pipeline stages, AND between execution units AND connect cores to shared cache banks. \smallskip

            \item \emp{System Interconnects:} connect processors (CMPs) to memory and IO

            \item \emp{I/O Interconnects}, usually bus e.g., PCI, connect various 
                    devices to the System Bus
            \item \emp{Inter-Systems Interconnects:} connect separate systems (chassis or boxes) \& include
                \begin{itemize}
                    \item \emph{SANs:} connect systems at very short distance
                    \item LANs, WANs (not interesting for us).
                \end  {itemize}

            \item Internet: global world-wide interconnect (not interesting for us).
\end  {itemize}

\end{frame}

%%%%%%%%%%%%%%%%%%%%%%%%%%%%%%%%%%%%%%%%%%%%
%%%%%%%%%%%%%%%%%%%%%%%%%%%%%%%%%%%%%%%%%%%%

\section{Technology Challenges/Constraints}

\begin{frame}[fragile]
	\tableofcontents[currentsection]
\end{frame}

\subsection{Power and Energy}

\begin{frame}[fragile,t]
\frametitle{Power}

\begin{itemize}
            \item \emp{\tt Total Power = Dynamic + Static (Leakage)}\\
                        $P_{dynamic} = \alpha C V^2 f$ \\
                        $P_{static}  = V I_{sub} \sim V e^{-k V_T / T}$

            \item Dynamic power favors parallel processing over higher clock rate
            \begin{itemize}
                \item $P_{dynamic} \sim F^3$
                \item replicate a uniprocessor 4 time $\Rightarrow$ 
                        $4\times$ speedup @ $4\times$ power
                \item increase clock 4 times $\Rightarrow$ 
                        $4\times$ speedup @ $64\times$ dynamic power!
            \end  {itemize}

            \item Static Power: because circuit leak whatever the frequency is, 
                    e.g., caches. It recently overtook dynamic power as major 
                    source of dissipation

            \item \emp{Power/Energy are Critical Problems}
            \begin{itemize}
                \item Power must be dissipated otherwise temperature
                        goes up \& affects performance, correctness and
                        may possibly destroy the circuit, short or long term. 
                \item Energy depends on power and speed. Costly \& 
                        many devices are battery-operated devices.

            \end  {itemize}

\end  {itemize}


\end{frame}


\section{Parallelism In Architectures: Vector \& Array Processors}

\begin{frame}[fragile]
	\tableofcontents[currentsection]
\end{frame}


\begin{frame}[fragile,t]
\frametitle{Scalar Multiprocessors}

\begin{itemize}
            \item Most Successful Architectures: Scalar Processor
                \begin{itemize}
                    \item instruction operates on scalar operands, 
                            e.g., {\tt ADD R1, R2, R3 $\Rightarrow$ R1 := R2 + R3}
                    \item execute multiple scalar instructions at a time:
                    \begin{itemize}
                        \item pipelining, 
                        \item superscalar
                        \item superpipelining
                        \item take advantage of instruction-level (intra-thread) 
                                parallelism (ILP).
                    \end  {itemize}
                \end  {itemize}

            \item Chip Multiprocessors (CMPs) exploits inter-thread parallelism:
                    \begin{itemize}
                        \item i.e., different threads running in parallel. 
                        \item multiple scalar processors running in parallel
                    \end  {itemize}
\end  {itemize}

%\begin{columns}
%\column{0.5\textwidth}
%\includegraphics[width=33ex]{Ch1Figs/DISK}
%\column{0.5\textwidth}
%\includegraphics[width=33ex]{Ch1Figs/Disk2}
%\end{columns}

\end{frame}

\begin{frame}[fragile,t]
\frametitle{Vector and Array Processors}

\begin{itemize}
            \item A typical vector instruction executes directly on vector operands,
                    e.g., {\tt VADD VR1,VR2,VR3 $\Rightarrow$ VR1[i]:=VR2[i]+VR3[i] 
                            $\forall$ i$\in\{1..N\}$}\bigskip

            \item Vector instructions are executed by pipelines or parallel arrays:
\end  {itemize}

\begin{columns}
\column{0.9\textwidth}
\includegraphics[width=65ex]{Ch1Figs/VectorMachine}
\column{0.1\textwidth}
\end{columns}

\end{frame}

\end{document}
