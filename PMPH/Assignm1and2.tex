\documentclass{beamer}

%% \documentclass[handout]{beamer}
%% % use this with the [handout] option to create handouts for the audience
%% \usepackage{pgfpages}
%% \pgfpagesuselayout{2 on 1}[a4paper,border shrink=5mm]

\mode<presentation>
{
  \usetheme{Diku}
% set this to your preferences:
  \setbeamercovered{invisible}
%  \setbeamercovered{transparent}
}

\usepackage{graphicx}
\usepackage{epic}

\usepackage{amsmath}
\usepackage{amssymb}
\usepackage{amsthm}

\newcommand{\basetop}[1]{\vtop{\vskip-1ex\hbox{#1}}}
\newcommand{\source}[1]{\let\thefootnote\relax\footnotetext{\scriptsize\textcolor{kugray1}{Source: #1}}}

% for coloured code citation in text:
\usepackage{fancyvrb}

%%%%%%%%%%%%%%%%%%%%%%%%%%%%%%%%%
%%%%%    code sections   %%%%%%%%
%%%%%%%%%%%%%%%%%%%%%%%%%%%%%%%%%

% code highlighting commands in own block
\DefineVerbatimEnvironment{code}{Verbatim}{fontsize=\scriptsize}
\DefineVerbatimEnvironment{icode}{Verbatim}{fontsize=\scriptsize}

% Fancy code with color commands:
\DefineVerbatimEnvironment{colorcode}%
        {Verbatim}{fontsize=\scriptsize,commandchars=\\\{\}}

%%%%%%%%%%%%%%%%%%%%%%%%%%%%%%%%%%
%%%%%    some coloring    %%%%%%%%

\definecolor{Red}{RGB}{220,50,10}
\definecolor{Blue}{RGB}{0,51,102}
\definecolor{Yellow}{RGB}{102,51,0}
\definecolor{Orange}{RGB}{178,36,36}
\definecolor{Grey}{RGB}{180,180,180}
\definecolor{Green}{RGB}{20,120,20}
\definecolor{Purple}{RGB}{160,50,100}
\newcommand{\red}[1]{\textcolor{Red}{{#1}}}
\newcommand{\blue}[1]{\textcolor{Blue}{{#1}}}
\newcommand{\yellow}[1]{\textcolor{Yellow}{{#1}}}
\newcommand{\orange}[1]{\textcolor{Orange}{{#1}}}
\newcommand{\grey}[1]{\textcolor{Grey}{{#1}}}
\newcommand{\green}[1]{\textcolor{Green}{{#1}}}
\newcommand{\purple}[1]{\textcolor{Purple}{{#1}}}




% use "DIKU green" from our color theme for \emph
\renewcommand{\emph}[1]{\textcolor{structure}{#1}}
% use some not-too-bright red for an \emp command
\definecolor{DikuRed}{RGB}{130,50,32}
\newcommand{\emp}[1]{\textcolor{DikuRed}{ #1}}
\definecolor{CosGreen}{RGB}{10,100,70}
\newcommand{\emphh}[1]{\textcolor{CosGreen}{ #1}}
\definecolor{CosBlue}{RGB}{55,111,122}
\newcommand{\emphb}[1]{\textcolor{CosBlue}{ #1}}
\definecolor{CosRed}{RGB}{253,1,1}
\newcommand{\empr}[1]{\textcolor{CosRed}{ #1}}

\newcommand{\mymath}[1]{$ #1 $}
\newcommand{\myindx}[1]{_{#1}}
\newcommand{\myindu}[1]{^{#1}}

\newcommand{\Fasto}{\textsc{Fasto}\xspace}


%%%%%%%%%%%%%%%%%%%%

\title[Assignments]{First and Second Assignments Discussion}

\author[C.~Oancea]{Cosmin E. Oancea {\tt cosmin.oancea@diku.dk}}

\institute{Department of Computer Science (DIKU)\\University of Copenhagen}


\date[Sept 2014]{September 2014 PMPH Lecture Notes}


\begin{document}

\titleslide

\begin{frame}[fragile,t]
\frametitle{Common Missunderstandings}

\begin{itemize}
    \item ``Task 4 from Assignment 1 is not well specified!''\\\pause
        \begin{itemize}
            \item It is intentionally underspecified because of the
            \item Context: Programming Models to Exploit Hardware Parallelism. 
            \item expecting \emp{best Work and Depth asymptotic AND flat parallelism.}  
        \end  {itemize}\medskip
    \item I apologize to the students whose code I am going to show,
    \item I only show it for didactic purposes, so that we all learn,
    \item I actually picked the ones that were close(r) to a correct solution,
    \item and I am sure that most of them were thought as a partial implementation...\medskip

    \item \alert{Do I have your permission to show some of them in didactic purposes?}

\end  {itemize}
\end{frame}

\begin{frame}[fragile,t]
\frametitle{But ... my Haskell Program Works (?)}

Haskell used just as a vehicle to reason about ``parallelism''
in terms of bulk, parallel primitive operators: 
map, reduce, scan, permute, write.\medskip

\begin{colorcode}[fontsize=\scriptsize]
 flatSparseMatVctMult flags mat x = 
    let vec_os  = \emph{map (\mymath{\backslash}n -> n-1) \$ scanInc (+) 0 flags}
        tmp_mat = \emph{segmScanInc (+) 0 flags \$ map (\mymath{\backslash}(col,val) -> x!!col * val) mat}
    in \alert{extract_result vec_os tmp_mat}
extract_result :: [Int] -> [a] -> [a]
extract_result flags mat = let 
    extract_result_inner [] [] _ = []
    extract_result_inner [x] [z] tmpres = tmpres++[z]
    extract_result_inner (x:y:xs) (z:zs) tmpres = 
        if x /= y
        then extract_result_inner (y:xs) zs (tmpres++[z])
        else extract_result_inner (y:xs) zs tmpres
    extract_result_inner _ _ _ = []
  in extract_result_inner flags mat []
\end{colorcode}
\pause

\alert{Remember: Amdhal's low is unforgiving}: if  
{\tt extract\_results} takes a third of sequential 
exec time, what is the maximal speedup?
\medskip

You might get partial marks, but little performance gain!


\end{frame}



\end{document}
