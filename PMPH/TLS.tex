\documentclass{beamer}

%% \documentclass[handout]{beamer}
%% % use this with the [handout] option to create handouts for the audience
%% \usepackage{pgfpages}
%% \pgfpagesuselayout{2 on 1}[a4paper,border shrink=5mm]

\mode<presentation>
{
  \usetheme{Diku}
% set this to your preferences:
  \setbeamercovered{invisible}
%  \setbeamercovered{transparent}
}

\usepackage{graphicx}
\usepackage{epic}

\usepackage{amsmath}
\usepackage{amssymb}
\usepackage{amsthm}

\newcommand{\basetop}[1]{\vtop{\vskip-1ex\hbox{#1}}}
\newcommand{\source}[1]{\let\thefootnote\relax\footnotetext{\scriptsize\textcolor{kugray1}{Source: #1}}}

% for coloured code citation in text:
\usepackage{fancyvrb}

%%%%%%%%%%%%%%%%%%%%%%%%%%%%%%%%%
%%%%%    code sections   %%%%%%%%
%%%%%%%%%%%%%%%%%%%%%%%%%%%%%%%%%

% code highlighting commands in own block
\DefineVerbatimEnvironment{code}{Verbatim}{fontsize=\scriptsize}
\DefineVerbatimEnvironment{icode}{Verbatim}{fontsize=\scriptsize}

% Fancy code with color commands:
\DefineVerbatimEnvironment{colorcode}%
        {Verbatim}{fontsize=\scriptsize,commandchars=\\\{\}}

%%%%%%%%%%%%%%%%%%%%%%%%%%%%%%%%%%
%%%%%    some coloring    %%%%%%%%

\definecolor{Red}{RGB}{220,50,10}
\definecolor{Blue}{RGB}{0,51,102}
\definecolor{Yellow}{RGB}{102,51,0}
\definecolor{Orange}{RGB}{178,36,36}
\definecolor{Grey}{RGB}{180,180,180}
\definecolor{Green}{RGB}{20,120,20}
\definecolor{Purple}{RGB}{160,50,100}
\newcommand{\red}[1]{\textcolor{Red}{{#1}}}
\newcommand{\blue}[1]{\textcolor{Blue}{{#1}}}
\newcommand{\yellow}[1]{\textcolor{Yellow}{{#1}}}
\newcommand{\orange}[1]{\textcolor{Orange}{{#1}}}
\newcommand{\grey}[1]{\textcolor{Grey}{{#1}}}
\newcommand{\green}[1]{\textcolor{Green}{{#1}}}
\newcommand{\purple}[1]{\textcolor{Purple}{{#1}}}




% use "DIKU green" from our color theme for \emph
\renewcommand{\emph}[1]{\textcolor{structure}{#1}}
% use some not-too-bright red for an \emp command
\definecolor{DikuRed}{RGB}{130,50,32}
\newcommand{\emp}[1]{\textcolor{DikuRed}{ #1}}
\definecolor{CosGreen}{RGB}{10,100,70}
\newcommand{\emphh}[1]{\textcolor{CosGreen}{ #1}}
\definecolor{CosBlue}{RGB}{55,111,122}
\newcommand{\emphb}[1]{\textcolor{CosBlue}{ #1}}
\definecolor{CosRed}{RGB}{253,1,1}
\newcommand{\empr}[1]{\textcolor{CosRed}{ #1}}

\newcommand{\mymath}[1]{$ #1 $}
\newcommand{\myindx}[1]{_{#1}}
\newcommand{\myindu}[1]{^{#1}}

\newcommand{\Fasto}{\textsc{Fasto}\xspace}


%%%%%%%%%%%%%%%%%%%%

\title[S-TLS]{Software Thread-Level Speculation (S-TLS)}

\author[C.~Oancea]{Cosmin E. Oancea\\{\tt cosmin.oancea@diku.dk}}

\institute{Department of Computer Science (DIKU)\\University of Copenhagen}


\date[Oct'14]{October 2014 PMPH Lecture Notes}


\begin{document}

\titleslide

\begin{frame}
\frametitle{Course Organization}

\begin{tabular}{lccccc}
W  & HARDWARE  & & SOFTWARE     & & LAB/CUDA \\\hline\hline
1 & Trends        &                         & List HOM     & & Intro \& Simple\\
  & Vector Machine & $\longleftarrow$ & (Map-Reduce) & & Map Programming\\\hline
%
2 & In Order & $\longrightarrow$ & VLIW Instr   & & Scan \&\\
  & Processor& $\longleftarrow$ & Scheduling   & & Reduce \\\hline
%
3 & \emphh{Cache}     & & Reasoning About     & & Sparse Vect\\
  & \emphh{Coherence} & & Parallelism   & & Matrix Mult\\\hline
%
4 & Interconnection & & Case Studies \&   & & Transpose \& Matrix\\
  & Networks        & & Optimizations   & & Matrix Mult\\\hline
%
5 & \emphh{Memory}      & & Optimising   & & Sorting \& Profiling \& \\
  & \emphh{Consistency} & & Locality     & & Mem Optimizations \\\hline
%
6 & \emphh{OoO, Spec}   & $\longleftarrow$ & \alert{Thread-Level}   & & Project \\
  & \emphh{Processor}   & & \alert{Speculation}    & & Work    \\\hline

%\framebox{Processor}       & & \framebox{Low-Level\\Optimizations}        & & \framebox{CUDA: Scan\\Reduce}\\
%$\downarrow$ && $\uparrow$ \\
%\framebox{\red Intermediate code generation} &$\longrightarrow$ & Intermediate code
\end{tabular}
\medskip
%\alert{Keywords: Reasoning, Tradeoffs, Common Case, }

Three narative threads: the path to complex \& good design: 
\begin{itemize}
    \item \emp{Design Space} tradeoffs, constraints, common case, trends.
    \item \emp{Reasoning}: from simple to complex, \emp{Applying Concepts}.
\end  {itemize}
\end{frame}



%%%%%%%% real content starts here %%%%%%%%%%

\begin{frame}[fragile,t]
  \frametitle{Motivation}

\emphh{So far one perfect-loop nest with affine accesses in shared memory}:
\begin{itemize}
    \item loop interchange,
    \item loop distribution,
    \item block tiling, e.g., matrix transposition, multiplication.
\end  {itemize}\bigskip

\begin{block}{Example: Loop Interchange Enhances Locality of Reference}
\begin{columns}
\column{0.47\textwidth}
\begin{colorcode}
// Bad locality both GPU \& CPU
\emphh{DOALL j = 1, N-1} // grid
  \emphh{DOALL i = 0, N-1} // block
    A[i,j] = sqrt(A[i,j] + B[i,j]);
  ENDDO
ENDDO
\end{colorcode}
\column{0.47\textwidth}
\begin{colorcode}
// Good locality both GPU \& GPU
\emphh{DOALL i = 0, N-1}
  \emphh{DOALL j = 1, N-1}    
    A[i,j] = sqrt(A[i,j] + B[i,j]);
  ENDDO
ENDDO
\end{colorcode}
\end{columns}
\end{block} 
 
\alert{But a program is a composition of loop nests \&\\ 
accesses are not always affine \&\\
how about communication in distributed programs (?!)}

\end{frame}


\begin{frame}[fragile]
	\tableofcontents
\end{frame}


\end{document}

