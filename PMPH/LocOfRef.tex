\documentclass{beamer}

%% \documentclass[handout]{beamer}
%% % use this with the [handout] option to create handouts for the audience
%% \usepackage{pgfpages}
%% \pgfpagesuselayout{2 on 1}[a4paper,border shrink=5mm]

\mode<presentation>
{
  \usetheme{Diku}
% set this to your preferences:
  \setbeamercovered{invisible}
%  \setbeamercovered{transparent}
}

\usepackage{graphicx}
\usepackage{epic}

\usepackage{amsmath}
\usepackage{amssymb}
\usepackage{amsthm}

\newcommand{\basetop}[1]{\vtop{\vskip-1ex\hbox{#1}}}
\newcommand{\source}[1]{\let\thefootnote\relax\footnotetext{\scriptsize\textcolor{kugray1}{Source: #1}}}

% for coloured code citation in text:
\usepackage{fancyvrb}

%%%%%%%%%%%%%%%%%%%%%%%%%%%%%%%%%
%%%%%    code sections   %%%%%%%%
%%%%%%%%%%%%%%%%%%%%%%%%%%%%%%%%%

% code highlighting commands in own block
\DefineVerbatimEnvironment{code}{Verbatim}{fontsize=\scriptsize}
\DefineVerbatimEnvironment{icode}{Verbatim}{fontsize=\scriptsize}

% Fancy code with color commands:
\DefineVerbatimEnvironment{colorcode}%
        {Verbatim}{fontsize=\scriptsize,commandchars=\\\{\}}

%%%%%%%%%%%%%%%%%%%%%%%%%%%%%%%%%%
%%%%%    some coloring    %%%%%%%%

\definecolor{Red}{RGB}{220,50,10}
\definecolor{Blue}{RGB}{0,51,102}
\definecolor{Yellow}{RGB}{102,51,0}
\definecolor{Orange}{RGB}{178,36,36}
\definecolor{Grey}{RGB}{180,180,180}
\definecolor{Green}{RGB}{20,120,20}
\definecolor{Purple}{RGB}{160,50,100}
\newcommand{\red}[1]{\textcolor{Red}{{#1}}}
\newcommand{\blue}[1]{\textcolor{Blue}{{#1}}}
\newcommand{\yellow}[1]{\textcolor{Yellow}{{#1}}}
\newcommand{\orange}[1]{\textcolor{Orange}{{#1}}}
\newcommand{\grey}[1]{\textcolor{Grey}{{#1}}}
\newcommand{\green}[1]{\textcolor{Green}{{#1}}}
\newcommand{\purple}[1]{\textcolor{Purple}{{#1}}}




% use "DIKU green" from our color theme for \emph
\renewcommand{\emph}[1]{\textcolor{structure}{#1}}
% use some not-too-bright red for an \emp command
\definecolor{DikuRed}{RGB}{130,50,32}
\newcommand{\emp}[1]{\textcolor{DikuRed}{ #1}}
\definecolor{CosGreen}{RGB}{10,100,70}
\newcommand{\emphh}[1]{\textcolor{CosGreen}{ #1}}
\definecolor{CosBlue}{RGB}{55,111,122}
\newcommand{\emphb}[1]{\textcolor{CosBlue}{ #1}}
\definecolor{CosRed}{RGB}{253,1,1}
\newcommand{\empr}[1]{\textcolor{CosRed}{ #1}}

\newcommand{\mymath}[1]{$ #1 $}
\newcommand{\myindx}[1]{_{#1}}
\newcommand{\myindu}[1]{^{#1}}

\newcommand{\Fasto}{\textsc{Fasto}\xspace}


%%%%%%%%%%%%%%%%%%%%

\title[Locality]{Optimising Locality of Reference}

\author[C.~Oancea]{Cosmin E. Oancea\\{\tt cosmin.oancea@diku.dk}}

\institute{Department of Computer Science (DIKU)\\University of Copenhagen}


\date[Sept 2014]{September 2014 PMPH Lecture Notes}


\begin{document}

\titleslide

\begin{frame}
\frametitle{Course Organization}

\begin{tabular}{lccccc}
W  & HARDWARE  & & SOFTWARE     & & LAB/CUDA \\\hline\hline
1 & Trends         &                         & List HOM     & & Intro \& Simple\\
  & Vector Machine & $\longleftarrow$ & (Map-Reduce) & & Map Programming\\\hline
%
2 & In Order & $\longrightarrow$ & VLIW Instr   & & Scan \&\\
  & Processor& $\longleftarrow$ & Scheduling   & & Reduce \\\hline
%
3 & Cache     & & Loop          & & Sparse Vect\\
  & Coherence & & Parallelism I & & Matrix Mult\\\hline
%
4 & Interconnection & & Case Studies \&   & & Transpose \& Matrix\\
  & Networks        & & Optimizations   & & Matrix Mult\\\hline
%
5 & Memory      & & \emp{Optimising}   & & Sorting \& Profiling \& \\
  & Consistency & & \emp{Locality}     & & Mem Optimizations \\\hline
%
6 & OoO, Spec   & & Thread-Level   & & Project \\
  & Processor   & & Speculation    & & Work    \\\hline

%\framebox{Processor}       & & \framebox{Low-Level\\Optimizations}        & & \framebox{CUDA: Scan\\Reduce}\\
%$\downarrow$ && $\uparrow$ \\
%\framebox{\red Intermediate code generation} &$\longrightarrow$ & Intermediate code
\end{tabular}
\medskip
%\alert{Keywords: Reasoning, Tradeoffs, Common Case, }

Three narative threads: the path to complex \& good design: 
\begin{itemize}
    \item \emp{Design Space} tradeoffs, constraints, common case, trends.
    \item \emp{Reasoning}: from simple to complex, \emp{Applying Concepts}.
\end  {itemize}
\end{frame}



%%%%%%%% real content starts here %%%%%%%%%%

\begin{frame}
  \frametitle{Motivation}

\begin{itemize}
    \item[+] So far we reasoned about how to parallelize a known algorithm
    \item[+] using a clean, functional approach, e.g., flattening, 
    \item[+] which provides work and depth guarantees,
    \item[\alert{-}] but does \alert{NOT} account for locality of reference.

\end  {itemize}\bigskip

\emp{Why do we have to look at imperative loops?}
\begin{itemize}    
    \item A lot of legacy sequential imperative code, C{\tt++}/Java/Fortran.\medskip
    \item Need to parallelize the implementation of unknown algorithm,\medskip
    \item Need to optimize parallelism, e.g., locality of reference requires subscript analysis. 
\end  {itemize}  

\end{frame}


\section{Direction-Vector Analysis}

\begin{frame}[fragile]
	\tableofcontents[currentsection]
\end{frame}


\begin{frame}[fragile,t]
  \frametitle{Problem Statement} % of CPU, Multicores, GPGPU

%[fontsize=\small]
\begin{block}{Three Loop Examples}
\begin{colorcode}
DO i = 1, N             DO i = 2, N                 DO i = 2, N
  DO j = 1, N             DO j = 2, N                 DO j = 1, N 
    A[j,i] = A[j,i] ..      A[j,i] = A[j-1,i-1]...        A[i,j] = A[i-1,j+1]...
  ENDDO                     B[j,i] = B[j-1,i]...      ENDDO
ENDDO                   ENDDO ENDDO                 ENDDO
\end{colorcode}
\end{block} 

Iterations are ordered {\em lexicographically}, i.e., in the order
they occur in the sequential execution, e.g., 
{\tt$\vec{k}=$(i=2,j=4) < $\vec{l}=$(i=3,j=3)}.

\bigskip

\begin{itemize}
    \item \emp{Which of the three loop nests is amenable to parallelization?}\smallskip
    \item Loop interchange is one of the most simple and useful code transformations,
            e.g., used to enhance locality of reference, parallel-loop granularity,
            and even to ``create'' parallelism.\smallskip
    \item \emp{In which loop nest is it safe to interchange the loops?}
\end{itemize}


\end{frame}

\end{document}

