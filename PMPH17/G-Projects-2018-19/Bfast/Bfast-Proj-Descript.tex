\documentclass[a4paper,11pt]{article}

\usepackage[top=2.5cm, bottom=2.5cm]{geometry}

\usepackage{multicol}
\usepackage{verbatim}
\usepackage{enumerate}
\usepackage{enumitem}

\usepackage{fancyvrb}

\usepackage{graphicx}
\usepackage{color}
\usepackage{subfigure}
%\usepackage{epsfig}
\usepackage{xspace}

\usepackage[utf8]{inputenc}
\usepackage[T1]{fontenc}
\usepackage{pslatex}
\usepackage[english]{babel}

\usepackage[colorlinks=true,linkcolor=black]{hyperref}

\usepackage{fancyhdr}%
\usepackage{lastpage}%
%
\pagestyle{fancy}%
\lhead{}%
\chead{}%
\rhead{}%
\cfoot{\thepage/\pageref{LastPage}}%
\renewcommand{\headrulewidth}{0in}%
%


%%%%%%%%%%%%%%%%%%%%%%%%%%%%%%%%%%
%%%%%  reviewer comments  %%%%%%%%
%%%%%%%%%%%%%%%%%%%%%%%%%%%%%%%%%%
\newcommand{\comm}[2]{{\sf \(\spadesuit\){\bf #1: }{\rm \sf #2}\(\spadesuit\)}}
\newcommand{\mcomm}[2]{\marginpar{\scriptsize \comm{#1}{#2}}}
%\renewcommand{\comm}[2]{}
%%%%%%%%%%%%%%%%%%%%%%%%%%%%%%%%%%
\newcommand{\jbcomment}[1]{\mcomm{JB}{#1}}
\newcommand{\cocomment}[1]{\mcomm{CO}{#1}}

% Fancy code with color commands:
\DefineVerbatimEnvironment{fancycode}%
        {Verbatim}{commandchars=\\\{\}}

%%%%%%%%%% The language %%%%%%%%%%
\newcommand{\fasto}{\textsc{Fasto}\xspace}
\newcommand{\mips}{\textsc{Mips}\xspace}
\newcommand{\mars}{\textsc{Mars}\xspace}
\newcommand{\soac}{\textsc{soac}\xspace}
\newcommand{\soacs}{\textsc{soac}s\xspace}

\begin{document}

%\setlength{\baselineskip}{0.95\baselineskip}

\begin{center}

{\LARGE Bfast: Translating Futhark Code Efficiently in CUDA}

\end{center}

\section*{Preamble}

This project refers to writing an efficient CUDA implementation
for a variation of the BFAST algorithm, which is one of the
state-of-the-art methods for break detection given satellite image
time series. For example BFAST can be used to detect environmental
changes, such as deforestization in the Amazonian jungle, etc.

The attached paper "Massively-Parallel Break Detection 
for Satellite Data" reports an implementation that assumes that
that all pixels hold valid data. This is somewhat unrealistic
because a certain area may be covered by clouds---the tested 
dataset is from an area located in Sahara, which is never clouded.

The attached
Futhark implementation is work in progress (hence do not distribute) 
and fixes the cloud issue by semantically ``filtering'' out the 
clouded pixels---encoded by NAN values--in the implementation of
various (mathematical) operations. Note that filtering does not
necessarily mean in this context the application of the {\tt filter} 
operator, but rather it means that the iteration space is padded
and {\tt NAN} values are ignored.

For example, the Futhark implementation of dotproduct is:

\begin{fancycode}
let dotprod [n] (xs: [n]f32) (ys: [n]f32): f32 =
    reduce (+) 0.0 (map2 (*) xs ys) 
\end{fancycode}

and the filtered implementation receives an additional flag array
in which the invalid pixels, marked with the flag-value {\tt false},
are ignored:

\begin{fancycode}
let dotprod_filt [n] (flgs: [n]bool) (xs: [n]f32) (ys: [n]f32) : f32 =
    f32.sum (map3 (\textbackslash{} flg x y -> if flg then x*y else 0) flgs xs ys)
\end{fancycode}

\section{Tasks}

Your task is to write a CUDA program, which is semantically equivalent
with the Futhark one, and which is as efficient as possible:
\begin{itemize}
    \item Please provide built-in validation for the provided Sahara 
            dataset---files {\tt sahara.in} and {\tt sahara.out} are the 
            reference input and result.
    \item Identify the opportunities for optimizing locality of reference,
        and apply related optimizations whenever is possible and profitable
        to do so, for example:
            \begin{itemize} 
            \item[1] make sure that all accesses to global memory are coalesced
                        (i.e., spatial locality),
            \item[2] identify the opportunities for one/two/three dimensional 
                tiling in shared memory and apply them whenever profitable;
            \item[3] identify opportunities in which a kernel loop can be 
                executed entirely (or mostly) in shared memory, for example
                in the case of matrix inversion (e.g., the {\tt gauss\_jordan} 
                Futhark function).
            \item[4] identify the cases in which a composition of {\tt scan}, 
                {\tt filter}, {\tt map} or {\tt scatter} operations can be
                performed in (fast) shared memory. For example, if the size
                of the scanned segment is less than 1024, it is much faster to
                perform the {\tt scan} within each block, then to perform
                a segmented scan across all elements in all blocks (why?). 
            \end{itemize}
    \item Try to report the impact of your optimizations both locally and
            globally.
        \begin{itemize}
            \item Locally means to separately test the impact of optimizations:
            for example what is the speedup of tiling matrix-vector multiplication, 
            with respect to its untiled version;
            \item Globally means to test the speedup generated by an optimization
            (e.g., tiling) at the level of the whole application runtime. 
        \end{itemize}
    \item Also please report your global performance, for example in terms of 
            percentage of the peak bandwidth of the system (and perhaps also in 
            terms of the sequential program --- could be the one generated 
            by futhark-c).
\end{itemize}

\section{Closing Remarks}

\begin{itemize}
    \item Please write a tidy report that puts emphasis on the parallelization
            and optimization strategy reasoning, and which explores and explains 
            the design space of potential solutions, and argues the strengths 
            and shortcomings of the one you have chosen.  For example present
            on representative example how optimizations such as tiling were
            applied, and why their application is safe.

    \item Please make sure your code validates at least on the provided datasets,
            and please compare the performance of your code against baseline 
            implementations in Futhark (or from elsewhere if you have other), 
            and comment on the potential differences between the observed
            performance and what you assumed it will be (based on the
            design-space exploration).
\end{itemize}

\bibliographystyle{unsrt}
\bibliography{CB12}
%\bibliography{GroupProj}


\end{document}
