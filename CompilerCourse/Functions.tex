\documentclass{beamer}

%% \documentclass[handout]{beamer}
%% % use this with the [handout] option to create handouts for the audience
%% \usepackage{pgfpages}
%% \pgfpagesuselayout{2 on 1}[a4paper,border shrink=5mm]

\mode<presentation>
{
  \usetheme{Diku}
% set this to your preferences:
  \setbeamercovered{invisible}
%  \setbeamercovered{transparent}
}

\usepackage{graphicx}
\usepackage{epic}

\usepackage{amsmath}
\usepackage{amssymb}
\usepackage{amsthm}

\newcommand{\basetop}[1]{\vtop{\vskip-1ex\hbox{#1}}}
\newcommand{\source}[1]{\let\thefootnote\relax\footnotetext{\scriptsize\textcolor{kugray1}{Source: #1}}}

% for coloured code citation in text:
\usepackage{fancyvrb}

%%%%%%%%%%%%%%%%%%%%%%%%%%%%%%%%%
%%%%%    code sections   %%%%%%%%
%%%%%%%%%%%%%%%%%%%%%%%%%%%%%%%%%

% code highlighting commands in own block
\DefineVerbatimEnvironment{code}{Verbatim}{fontsize=\scriptsize}
\DefineVerbatimEnvironment{icode}{Verbatim}{fontsize=\scriptsize}

% Fancy code with color commands:
\DefineVerbatimEnvironment{colorcode}%
        {Verbatim}{fontsize=\scriptsize,commandchars=\\\{\}}

%%%%%%%%%%%%%%%%%%%%%%%%%%%%%%%%%%
%%%%%    some coloring    %%%%%%%%

\definecolor{Red}{RGB}{220,50,10}
\definecolor{Blue}{RGB}{0,51,102}
\definecolor{Yellow}{RGB}{102,51,0}
\definecolor{Orange}{RGB}{178,36,36}
\definecolor{Grey}{RGB}{180,180,180}
\definecolor{Green}{RGB}{20,120,20}
\definecolor{Purple}{RGB}{160,50,100}
\newcommand{\red}[1]{\textcolor{Red}{{#1}}}
\newcommand{\blue}[1]{\textcolor{Blue}{{#1}}}
\newcommand{\yellow}[1]{\textcolor{Yellow}{{#1}}}
\newcommand{\orange}[1]{\textcolor{Orange}{{#1}}}
\newcommand{\grey}[1]{\textcolor{Grey}{{#1}}}
\newcommand{\green}[1]{\textcolor{Green}{{#1}}}
\newcommand{\purple}[1]{\textcolor{Purple}{{#1}}}




% use "DIKU green" from our color theme for \emph
\renewcommand{\emph}[1]{\textcolor{structure}{#1}}
% use some not-too-bright red for an \emp command
\definecolor{DikuRed}{RGB}{130,50,32}
\newcommand{\emp}[1]{\textcolor{DikuRed}{ #1}}
\definecolor{CosGreen}{RGB}{10,100,70}
\newcommand{\emphh}[1]{\textcolor{CosGreen}{ #1}}
\definecolor{CosBlue}{RGB}{55,111,122}
\newcommand{\emphb}[1]{\textcolor{CosBlue}{ #1}}
\definecolor{CosRed}{RGB}{253,1,1}
\newcommand{\empr}[1]{\textcolor{CosRed}{ #1}}

\newcommand{\mymath}[1]{$ #1 $}
\newcommand{\myindx}[1]{_{#1}}
\newcommand{\myindu}[1]{^{#1}}

\newcommand{\Fasto}{\textsc{Fasto}\xspace}


%%%%%%%%%%%%%%%%%%%%

\title[Functions]{Machine-Code Generation for Functions}

\author[C.~Oancea]{Cosmin E. Oancea\\{\tt cosmin.oancea@diku.dk}}

\institute{Department of Computer Science (DIKU)\\University of Copenhagen}


\date[December 2012]{December 2012 Compiler Lecture Notes}


\begin{document}

\titleslide

\begin{frame}
\frametitle{Structure of a Compiler}

\begin{tabular}{ccc}
Program text&&\\
$\downarrow$ &&\\
\framebox{Lexical analysis} && Binary machine code\\
$\downarrow$ && $\uparrow$ \\
Symbol sequence && \textcolor{gray}{\framebox{Assembly and linking}} \\
$\downarrow$ && $\uparrow$ \\
\framebox{Syntax analysis} && Ditto with named registers\\
$\downarrow$ && $\uparrow$ \\
Syntax tree && \framebox{Register allocation} \\
$\downarrow$ && $\uparrow$ \\
\framebox{Type Checking} && Symbolic machine code\\
$\downarrow$ &&  $\uparrow$ \\
Syntax tree  && \framebox{\red{Machine code generation}} \\
$\downarrow$ && $\uparrow$ \\
\framebox{Intermediate code generation} &$\longrightarrow$ & Intermediate code
\end{tabular}

\end{frame}


\begin{frame}[fragile]
	\tableofcontents
\end{frame}

%%%%%%%%%%%%%%%%%%%%%%%%%%%%%%%%%%%%%%%%%%%%%%%%%%%%%%%%%%%%%%%%%%%%%%
%%%%%%%%%%%%%%%%%%%%%%%%%%%%%%%%%%%%%%%%%%%%%%%%%%%%%%%%%%%%%%%%%%%%%%
\section{Problem Statement and Terminology}

\begin{frame}[fragile,t]
   \frametitle{Problem Statement}

\bigskip

So far we have generated code for a single function:

\bigskip

\begin{itemize}

    \item A function call is simply translated to a {\tt CALL} instruction.\bigskip

    \item Register allocation is performed on a single-function body.\bigskip

    \item We have assumed that all variables are local.\bigskip

    \item We have assumed that the function's parameters and result are
            passed via  named variables (symbolic registers).

\end{itemize}

\bigskip

\emp{How to implement these in the machine language?}

\end{frame}





\begin{frame}[fragile,t]
   \frametitle{Call Stack}

\bigskip

We use a stack to store the information that connects the 
caller to the callee when a function call occurs:


\bigskip

\begin{itemize}

    \item \emp{The return address} is stored on the stack.\bigskip

    \item \emp{The registers' content} is stored on the stack before the call and
            is restored (in registers) after the call.\bigskip

    \item \emp{Parameters and return value} are also passed on the stack.\bigskip

    \item \emp{Spilled variables} are also stored on the stack.\bigskip

    \item \emp{Local arrays/records} are also typically allocated on the stack.\bigskip

    \item Finally, \emp{non-local variables} can be allocated on the stack.

\end{itemize}


\end{frame}




\begin{frame}[fragile,t]
   \frametitle{Activation Records}

\bigskip

Each function allocates a piece of memory on the stack to keep
associated information. This piece of storage is called 
function's \emp{activation record} or \emp{frame}.


\bigskip


The hardware / operating system will dictate a calling convention
that would standardize \emph{\em the layout of the activation record} 
(so that we can call function across different compilers and languages).


\bigskip


However, some languages use extended calling conventions, such that
only ''simple'' functions can be called from other languages, i.e.,
foreign-function interface.

\end{frame}



%%%%%%%%%%%%%%%%%%%%%%%%%%%%%%%%%%%%%%%%%%%%%%%%%%%%%%%%%%%%%%%%%%%%%%
%%%%%%%%%%%%%%%%%%%%%%%%%%%%%%%%%%%%%%%%%%%%%%%%%%%%%%%%%%%%%%%%%%%%%%
\section{Caller-Saves Strategy}

\begin{frame}[fragile]
	\tableofcontents[currentsection]
\end{frame}


\begin{frame}
\frametitle{Caller-Saves: Activation-Record Layout}

\emp{The caller does the work of saving and restoring registers.}

\renewcommand{\arraystretch}{0.9}
\begin{center}
\begin{tabular}{r|p{17em}|}

& $\cdots$ \\

& Next activation records \\\hline

 & Space for storing local variables for spill or preservation across
function calls \\\cline{2-2}

 & Remaining incoming parameters  \\\cline{2-2}

 & First incoming parameter / return value  \\\cline{2-2}

FP $\longrightarrow$ & Return address \\\hline
& Previous activation records \\
& $\cdots$
\end{tabular}
\end{center}


\emp{The frame pointer, FP,} indicates the beginning of the activation
record.\smallskip

With this layout, \emph{the stack grows up in memory}.

\end{frame}


\begin{frame}
\frametitle{Prologue, Epilogue and Call Sequence}

The code of a function starts with a \emph{\em prologue}, that (i) retrieves
parameters from the stack and places them in variables (registers),
and (ii) \alert{may} save the registers to be preserved.

\bigskip

The code of a function ends with an \blue{\em epilogue} that places the return
value back on stack, and \alert{may} restore in registers the values saved in
the {\em prologue}, and then returns control to the calling function.

\bigskip

A CALL instruction is replaced with a \emp{\em call sequence}, which places
arguments on stack, saves the registers to be preserved, saves the
return address, calls the function and moves the returned value from
the stack to a variable (register), restores the saved registers.

\bigskip

What the {\em prologue, epilogue and call sequence} do exactly depends on
the calling convention.


\end{frame}



\begin{frame}
\frametitle{Caller-Saves: Prologue and Epilogue}

Assume a function with name {\it function-name} and parameters
$parameter_1 \ldots parameter_n$ . 
The result is calculated in variable $result$.\smallskip

We assume that the caller saves the registers: {\em caller-saves}.

\vspace{1ex}

\renewcommand{\arraystretch}{0.9}
\[\begin{array}{rl}

\mbox{Prologue} & \left\{
{\begin{array}{l}
{\tt LABEL}~\mbox{\it function-name} \\
parameter_1 := M[FP+4] \\
\cdots \\
parameter_n := M[FP+4*n]
\end{array}}
\right. \\
\\
& ~~~\mbox{\it code for the function body} \\
\\
\mbox{Epilogue} & \left\{
{\begin{array}{l}
M[FP+4] := result \\
{\tt GOTO}~M[FP]
\end{array}}
\right.
\end{array}\]

\vspace{1ex}

We used here {\sc il} instructions, but typically, the prologue, epilogue
and call sequence are introduced directly in machine language.

\end{frame}




\begin{frame}
\frametitle{Caller-Saves: Call Sequence}

Consider call $x := {\tt CALL}~f(a_1,\ldots,a_n)$.

Assume that $RO \ldots Rk$ are used for local variables. 

$framesize$ framesize is the {\em size} of the current activation record.


\vspace{0.5ex}

\renewcommand{\arraystretch}{0.87}
\[\begin{array}{l}
M[FP+4*m+4] := R0 \\
\cdots \\
M[FP+4*m+4*(k+1)] := Rk \\
\emp{FP := FP + framesize} \\
M[FP + 4] := a_1 \\
\cdots \\
M[FP + 4*n] := a_n \\
M[FP] := returnaddress \\
{\tt GOTO}~f \\
{\tt LABEL}~returnaddress \\
x := M[FP+4] \\
\emp{FP := FP - framesize} \\
R0 := M[FP+4*m+4] \\
\cdots \\
Rk := M[FP+4*m+4*(k+1)]
\end{array}\]

\end{frame}



%%%%%%%%%%%%%%%%%%%%%%%%%%%%%%%%%%%%%%%%%%%%%%%%%%%%%%%%%%%%%%%%%%%%%%
%%%%%%%%%%%%%%%%%%%%%%%%%%%%%%%%%%%%%%%%%%%%%%%%%%%%%%%%%%%%%%%%%%%%%%
\section{Callee-Saves Strategy}

\begin{frame}[fragile]
	\tableofcontents[currentsection]
\end{frame}


\begin{frame}
\frametitle{Callee-Saves: Activation Records}

\emph{The callee does all the work of saving and restoring registers.}


\renewcommand{\arraystretch}{0.9}

\begin{center}
\begin{tabular}{r|p{17em}|}

& $\cdots$ \\

& Next activation records \\\hline

 & Space for storing local variables for spill \\\cline{2-2}

 & Space for storing registers that need to be preserved \\\cline{2-2}

 & Remaining incoming parameters  \\\cline{2-2}

 & First incoming parameter / return value  \\\cline{2-2}

FP $\longrightarrow$ & Return address \\\hline
& Previous activation records \\
& $\cdots$
\end{tabular}
\end{center}

\emp{Difference: separate space for {\em saved} and  {\em spill} registers.}

\end{frame}



\begin{frame}
\frametitle{Callee-Saves: Prologue and Epilogue}

\renewcommand{\arraystretch}{0.9}

\[\begin{array}{rl}

\mbox{Prologue} & \left\{
{\begin{array}{l}
{\tt LABEL}~\mbox{\it function-name} \\
M[FP+4*n+4] := R0 \\
\cdots \\
M[FP+4*n+4*(k+1)] := Rk \\
parameter_1 := M[FP+4] \\
\cdots \\
parameter_n := M[FP+4*n]
\end{array}}
\right. \\
\\
& ~~~\mbox{\it code for the function body} \\
\\
\mbox{Epilogue} & \left\{
{\begin{array}{l}
M[FP+4] := result \\
R0 := M[FP+4*n+4] \\
\cdots \\
Rk := M[FP+4*n+4*(k+1)] \\
{\tt GOTO}~M[FP]
\end{array}}
\right.
\end{array}\]

\emp{Difference: $R0\ldots Rk$ are saved in {\em prologue} and restored in {\em epilogue}.}

\end{frame}


\begin{frame}
\frametitle{Callee-Saves: Call Sequence}

\renewcommand{\arraystretch}{0.9}

\[\begin{array}{l}
FP := FP + framesize \\
M[FP + 4] := a_1 \\
\cdots \\
M[FP + 4*n] := a_n \\
M[FP] := returnaddress \\
{\tt GOTO}~f \\
{\tt LABEL}~returnaddress \\
x := M[FP+4] \\
FP := FP - framesize
\end{array}\]

\bigskip

\emp{Difference: $R0\ldots Rk$ are not stored here.}

\end{frame}



\begin{frame}
\frametitle{Caller-Saves vs. Callee-Saves}

\bigskip

So far, no big difference, but:\pause\bigskip

\begin{itemize}

\item \emph{Caller-saves} need only save the registers containing live variables.\smallskip

\item \emp{Callee-saves} need only save the registers that are used in the
            function's body.

\end{itemize}

\bigskip

Can use a mixed strategy: some registers are \emph{caller-saves} and others
are \emp{callee-saves}.

\end{frame}


%%%%%%%%%%%%%%%%%%%%%%%%%%%%%%%%%%%%%%%%%%%%%%%%%%%%%%%%%%%%%%%%%%%%%%
%%%%%%%%%%%%%%%%%%%%%%%%%%%%%%%%%%%%%%%%%%%%%%%%%%%%%%%%%%%%%%%%%%%%%%
\section{Mixed Strategy (Caller + Callee Saves)}

\begin{frame}[fragile]
	\tableofcontents[currentsection]
\end{frame}


\begin{frame}
\frametitle{Use of Registers for Parameter Passing}

If parameters are passed on the stack, they must be transfered from
registers to stack and shortly after from stack to registers again.

\bigskip

The idea is to transfer some parameters and return value via registers:\smallskip

\begin{itemize}

\item Subset of caller-saves registers (typically $4$-$8$), used for parameter
        transfer (to be rarely preserved across the function call).\smallskip

\item A caller-saves register (often the same) is used for the result.\smallskip


\item The remaining parameters are transferred on the stack as shown.\smallskip

\item Often the return address is also transferred in a register.

\end{itemize}

\end{frame}



\begin{frame}
\frametitle{Typical Register Breakdown}

With a $16$-register processor:

\renewcommand{\arraystretch}{0.9}

\begin{center}
\begin{tabular}{|c|l|l|}\hline
Register & Saved by & Used for \\\hline
0 & caller & parameter 1 / result / local variable\\
1-3 & caller & parameters 2 - 4 / local variables \\
4-12 & callee & local variables \\
13 & caller & temporary storage (unused by register allocator) \\
14 & callee & FP \\
15 & callee & return address \\\hline
\end{tabular}
\end{center}

Typically there are more callee-saves than caller-saves registers.


\end{frame}


\begin{frame}
\frametitle{Activation Records for Register-Passed Parameters}

\renewcommand{\arraystretch}{0.9}

\begin{center}
\begin{tabular}{r|p{17em}|}

& $\cdots$ \\

& Next activation records \\\hline

 & Space for storing local variables for spill and for storing
live variables allocated to caller-saves registers across function
calls \\\cline{2-2}

 & Space for storing callee-saves registers that are used in the body
\\\cline{2-2}

 & Incoming parameters in excess of four \\\cline{2-2}

FP $\longrightarrow$ & Return address \\\hline & Previous activation
records \\ & $\cdots$
\end{tabular}
\end{center}

\end{frame}







\begin{frame}
\frametitle{Prologue and Epilogue for Register-Passed Params}

\renewcommand{\arraystretch}{0.85}

\[\begin{array}{rl}
\mbox{Prologue} & \left\{
{\begin{array}{l}
{\tt LABEL}~\mbox{\it function-name} \\
M[FP+\mbox{\it offset}_{R4}] := R4 ~~~~~~~~\mbox{(if used in body)} \\
\cdots \\
M[FP+\mbox{\it offset}_{R12}] := R12 ~~~~~~\mbox{(if used in body)} \\
M[FP] := R15 ~~~~~~\mbox{(if used in body)} \\
parameter_1 := R0\\
parameter_2 := R1\\
parameter_3 := R2\\
parameter_4 := R3\\
parameter_5 := M[FP+4] \\
\cdots \\
parameter_n := M[FP+4*(n-4)]
\end{array}}
\right. \\[0.5ex]

& ~~~~\mbox{\it code for the function body} \\[0.5ex]

\mbox{Epilogue} & \left\{
{\begin{array}{l}
R0 := result \\
R4 := M[FP+\mbox{\it offset}_{R4}] ~~~~~~~~\mbox{(if used in body)}  \\
\cdots \\
R12 := M[FP+\mbox{\it offset}_{R12}] ~~~~~~\mbox{(if used in body)} \\
R15 := M[FP] ~~~~~~\mbox{(if used in body)} \\
{\tt GOTO}~R15
\end{array}}
\right.
\end{array}\]

\end{frame}

\begin{frame}
\frametitle{Calling Sequence for Register-Passed Parameters}

\renewcommand{\arraystretch}{0.88}
\[\begin{array}{l}
M[FP+\mbox{\it offset}_{live_1}] := live_1  ~~~~~~\mbox{(if allocated to a
caller-saves register)} \\
\cdots \\
M[FP+\mbox{\it offset}_{live_k}] := live_k ~~~~~~\mbox{(if allocated to a
caller-saves register)} \\
FP := FP + framesize \\
R0 := a_1 \\
\cdots \\
R3 := a_4 \\
M[FP + 4] := a_5 \\
\cdots \\
M[FP + 4*(n-4)] := a_n \\
R15 := returnaddress \\
{\tt GOTO}~f \\
{\tt LABEL}~returnaddress \\
x := R0 \\
FP := FP - framesize \\
live_1 := M[FP+\mbox{\it offset}_{live_1}] ~~~~~~\mbox{(if allocated to a
caller-saves register)} \\
\cdots \\
live_k := M[FP+\mbox{\it offset}_{live_k}] ~~~~~~\mbox{(if allocated to a
caller-saves register)}
\end{array}\]

\end{frame}



\begin{frame}
\frametitle{Interaction with Register Allocator}

Register Allocation Can:\smallskip

\begin{itemize}
\item Preferably place the variables that are not live after the function
        call in caller-saves registers (so that they are not saved).\smallskip

\item Determine which caller-saves registers need to be saved by the
        caller before the function call.\smallskip\pause

\item Preferably place the variables that are live after the function call
        in callee-saves registers (i.e., the called function might not save
        them if those registers are not used).\smallskip

\item Determine which callee-saves register need to be saved by the
        callee, i.e., used in the callee's body.\smallskip


\item Eliminate the unnecessary copying of data to and from local vars.

\end{itemize}

\bigskip

The strategy used most often: \pause use caller-saves only for the variables
that are not live after the function call, hence they need not be saved.\\\smallskip
This simplifies \emp{the call sequence}.

\end{frame}




\begin{frame}
\frametitle{Typical Function-Code-Generation Strategy}

\begin{enumerate}

\item Generate code for the {\em function's body} using \emph{symbolic registers}
        for \emph{named variables}, but using \emp{numbered registers} in the call
        sequence for \emp{parameters}.\smallskip

\item  Add the prologue and epilogue code for moving \emp{numbered registers} 
            (and stack locations), i.e., \emp{parameters}, into 
            \emph{symbolic registers}, i.e., \emph{named variables} 
            (and the opposite for the result).\smallskip


\item Call the register allocator on this expanded function body.
        \begin{itemize}
            \item {\tt RegAlloc} is aware of the register division: caller vs callee saves,
            \item Allocates live-across-call variables only in callee-saves regs,
            \item Finds both the set of used callee-saves regs and of spilled vars.\smallskip
        \end{itemize}

\item Add to the prologue and epilogue the code for saving/restoring
        the callee-saves regs that {\tt RegAlloc} said to have been used 
        in the extended function body $+$ updating \textsc{FP} 
        (including space for saved regs and spilled vars).\smallskip


\item Add a function label at prologue's start and a jump at then end.

\end{enumerate}
\end{frame}

%%%%%%%%%%%%%%%%%%%%%%%%%%%%%%%%%%%%%%%%%%%%%%%%%%%%%%%%%%%%%%%%%%%%%%
%%%%%%%%%%%%%%%%%%%%%%%%%%%%%%%%%%%%%%%%%%%%%%%%%%%%%%%%%%%%%%%%%%%%%%
\section{Global Variables, Call by Reference, Aliasing}

\begin{frame}[fragile]
	\tableofcontents[currentsection]
\end{frame}

\begin{frame}
\frametitle{Treating Global Variables}

\emp{\em Global variables}: allocated in memory at statically-known addresses.\smallskip

Generated reading/writing code is similar to that of \emp{spilled variables}:\smallskip

\[\begin{array}{l}
x := M[address_{x}] \\
\mbox{instruction that uses}~ x\\
\\
x := \mbox{the value to be stored in}~x \\
M[address_{x}] := x
\end{array}\]

Various temporary variables (registers) may be used.\smallskip

If a global variable is frequently used in a function, then copy-into a
register in the beginning, and copy-out to memory at the end.\smallskip

Copy-in/out across fun calls $+$ extra care in the presence of \emp{aliasing}.

\bigskip

\emph{Morale: Use global variables sparingly, local variables are preferred.}

\end{frame}




\begin{frame}
\frametitle{Call By Reference}

\bigskip

The update to a function parameter is visible after the return point.
Also applies to fields of records, unions, array's elements, etc.\bigskip


Code generation typically treats such parameters as \emp{pointers}.\bigskip

Call by reference give rise to  {\em aliasing}.\bigskip

Morale: cheaper to update a variable via the returned value, rather than 
passing it by reference: \emp{{\tt x~=~f(x);} is cheaper than {\tt f(\&x);}.}

\end{frame}

\begin{frame}
\frametitle{Aliasing}

\emp{\em Aliasing}: if the same memory location can be accessed via two
different named variables.\bigskip


This can occur when the language allows references, either:\smallskip

\begin{itemize}

\item between two references pointing to the same place,\smallskip

\item between a global variable and a reference to this.\smallskip

\end{itemize}

\bigskip

If two names may alias, then:\smallskip

\begin{itemize}

\item before reading from one save the other to memory,\smallskip

\item if writing into one, then read the other again from memory
        before using it.\smallskip

\end{itemize}


Can be sometimes optimized be means of {\em aliasing analysis}.


\end{frame}


%%%%%%%%%%%%%%%%%%%%%%%%%%%%%%%%%%%%%%%%%%%%%%%%%%%%%%%%%%%%%%%%%%%%%%
%%%%%%%%%%%%%%%%%%%%%%%%%%%%%%%%%%%%%%%%%%%%%%%%%%%%%%%%%%%%%%%%%%%%%%


\end{document}
