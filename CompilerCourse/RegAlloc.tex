\documentclass{beamer}

%% \documentclass[handout]{beamer}
%% % use this with the [handout] option to create handouts for the audience
%% \usepackage{pgfpages}
%% \pgfpagesuselayout{2 on 1}[a4paper,border shrink=5mm]

\mode<presentation>
{
  \usetheme{Diku}
% set this to your preferences:
  \setbeamercovered{invisible}
%  \setbeamercovered{transparent}
}

\usepackage{graphicx}
\usepackage{epic}

\usepackage{amsmath}
\usepackage{amssymb}
\usepackage{amsthm}

\newcommand{\basetop}[1]{\vtop{\vskip-1ex\hbox{#1}}}
\newcommand{\source}[1]{\let\thefootnote\relax\footnotetext{\scriptsize\textcolor{kugray1}{Source: #1}}}

% for coloured code citation in text:
\usepackage{fancyvrb}

%%%%%%%%%%%%%%%%%%%%%%%%%%%%%%%%%
%%%%%    code sections   %%%%%%%%
%%%%%%%%%%%%%%%%%%%%%%%%%%%%%%%%%

% code highlighting commands in own block
\DefineVerbatimEnvironment{code}{Verbatim}{fontsize=\scriptsize}
\DefineVerbatimEnvironment{icode}{Verbatim}{fontsize=\scriptsize}

% Fancy code with color commands:
\DefineVerbatimEnvironment{colorcode}%
        {Verbatim}{fontsize=\scriptsize,commandchars=\\\{\}}

%%%%%%%%%%%%%%%%%%%%%%%%%%%%%%%%%%
%%%%%    some coloring    %%%%%%%%

\definecolor{Red}{RGB}{220,50,10}
\definecolor{Blue}{RGB}{0,51,102}
\definecolor{Yellow}{RGB}{102,51,0}
\definecolor{Orange}{RGB}{178,36,36}
\definecolor{Grey}{RGB}{180,180,180}
\definecolor{Green}{RGB}{20,120,20}
\definecolor{Purple}{RGB}{160,50,100}
\newcommand{\red}[1]{\textcolor{Red}{{#1}}}
\newcommand{\blue}[1]{\textcolor{Blue}{{#1}}}
\newcommand{\yellow}[1]{\textcolor{Yellow}{{#1}}}
\newcommand{\orange}[1]{\textcolor{Orange}{{#1}}}
\newcommand{\grey}[1]{\textcolor{Grey}{{#1}}}
\newcommand{\green}[1]{\textcolor{Green}{{#1}}}
\newcommand{\purple}[1]{\textcolor{Purple}{{#1}}}




% use "DIKU green" from our color theme for \emph
\renewcommand{\emph}[1]{\textcolor{structure}{#1}}
% use some not-too-bright red for an \emp command
\definecolor{DikuRed}{RGB}{130,50,32}
\newcommand{\emp}[1]{\textcolor{DikuRed}{ #1}}
\definecolor{CosGreen}{RGB}{10,100,70}
\newcommand{\emphh}[1]{\textcolor{CosGreen}{ #1}}
\definecolor{CosBlue}{RGB}{55,111,122}
\newcommand{\emphb}[1]{\textcolor{CosBlue}{ #1}}
\definecolor{CosRed}{RGB}{253,1,1}
\newcommand{\empr}[1]{\textcolor{CosRed}{ #1}}

\newcommand{\mymath}[1]{$ #1 $}
\newcommand{\myindx}[1]{_{#1}}
\newcommand{\myindu}[1]{^{#1}}

\newcommand{\Fasto}{\textsc{Fasto}\xspace}


%%%%%%%%%%%%%%%%%%%%

\title[Register Allocation]{Liveness Analysis and Register Allocation}

\author[C.~Oancea]{Cosmin E. Oancea\\{\tt cosmin.oancea@diku.dk}}

\institute{Department of Computer Science (DIKU)\\University of Copenhagen}


\date[December 2012]{December 2012 Compiler Lecture Notes}


\begin{document}

\titleslide

\begin{frame}
\frametitle{Structure of a Compiler}

\begin{tabular}{ccc}
Program text&&\\
$\downarrow$ &&\\
\framebox{Lexical analysis} && Binary machine code\\
$\downarrow$ && $\uparrow$ \\
Symbol sequence && \textcolor{gray}{\framebox{Assembly and linking}} \\
$\downarrow$ && $\uparrow$ \\
\framebox{Syntax analysis} && Ditto with named registers\\
$\downarrow$ && $\uparrow$ \\
Syntax tree && \framebox{\red{Register allocation}} \\
$\downarrow$ && $\uparrow$ \\
\framebox{Type Checking} && Symbolic machine code\\
$\downarrow$ &&  $\uparrow$ \\
Syntax tree  && \framebox{Machine code generation} \\
$\downarrow$ && $\uparrow$ \\
\framebox{Intermediate code generation} &$\longrightarrow$ & Intermediate code
\end{tabular}

\end{frame}


\begin{frame}[fragile]
	\tableofcontents
\end{frame}

%%%%%%%%%%%%%%%%%%%%%%%%%%%%%%%%%%%%%%%%%%%%%%%%%%%%%%%%%%%%%%%%%%%%%%
%%%%%%%%%%%%%%%%%%%%%%%%%%%%%%%%%%%%%%%%%%%%%%%%%%%%%%%%%%%%%%%%%%%%%%
\section{Problem Statement and Intuition}

\begin{frame}[fragile,t]
   \frametitle{Problem Statement}

\bigskip


\emp{Processors have a limited number of registers:}

\bigskip

\begin{description}

    \item[X86:] $8$ (integer) registers,\bigskip

    \item[ARM:] $16$ (integer) registers,\bigskip

    \item[MIPS:] $31$ (integer) registers.\bigskip

\end{description}

In addition, $3-4$ special-purpose registers (can't hold variables).

\bigskip
\bigskip

\emp{Solution:}

\bigskip

\begin{itemize}

    \item Whenever possible, let several variables share the same register,\bigskip

    \item If there are still variables that cannot be mapped to a register,
            store them in memory.

\end{itemize}


\end{frame}



\begin{frame}[fragile,t]
   \frametitle{Where to Implement Register Allocation?}

\bigskip

Two possibilities: at {\sc il} or at machine-language level. \alert{Pro/Cons?}

\pause
\smallskip

\begin{itemize}

    \item \emph{{\sc il} Level:}

        \begin{description}
            \item[+] \emph{Can be shared between multiple architectures
                    (parameterized on the number of registers).}
            \item[-] \emp{Translation to machine code can introduce/remove
                        intermediate results.}
        \end{description}\bigskip


    \item \emph{Machine-Code Level:}

        \begin{description}
            \item[+] \emph{Accurate, near-optimal mapping.}
            \item[-] \emp{Implemented for every architecture, no code reuse.}
        \end{description}

\end{itemize}

\bigskip

We show register allocation at {\sc il} level. Similar for machine code.

\end{frame}


\begin{frame}[fragile,t]
   \frametitle{Register-Allocation Scope}

\bigskip

\begin{itemize}

    \item \emph{Code Sequence Without Jumps:}

        \begin{description}
            \item[+] \emph{Simple.}
            \item[-] \emp{A variable is saved to memory when jumps occur.}
        \end{description}\bigskip


    \item \emph{Procedure/Function Level:}

        \begin{description}
            \item[+] \emph{Variables can still be in registers even across jumps.}
            \item[-] \emp{A bit more complicated.}
            \item[-] \emp{Variables saved to memory before function calls.}
        \end{description}

    \item \emph{Module/Program Level:}

        \begin{description}
            \item[+] \emph{Sometimes variables can still be hold in registers
                        across function calls (but not always: recursion).}
            \item[-] \emp{More complicated alg of higher time complexity. }
        \end{description}

\end{itemize}

\bigskip

Most compilers implement register allocation at function level.

\end{frame}



\begin{frame}[fragile,t]
   \frametitle{When Can Two Variables Share a Register?}

\bigskip
\bigskip

\begin{description}

    \item[Intuition:] Two vars can share a register if the two variables do not
                        have overlapping \emph{periods of use}.\bigskip

    \item[Period of Use:] From var's first assignment to the last use of the var.
                A variable can have several periods of use (\emph{\em live ranges}).\bigskip

    \item[{\em Liveness:}] If a variable's value may be used on the continuation of
                            an execution path passing through program point \textsc{PP},
                            then the variable is {\em live} at \textsc{PP}. 
                            Otherwise: {\em dead} at \textsc{PP}.\bigskip

\end{description}

\end{frame}



%%%%%%%%%%%%%%%%%%%%%%%%%%%%%%%%%%%%%%%%%%%%%%%%%%%%%%%%%%%%%%%%%%%%%%
%%%%%%%%%%%%%%%%%%%%%%%%%%%%%%%%%%%%%%%%%%%%%%%%%%%%%%%%%%%%%%%%%%%%%%
\section{Liveness-Analysis Preliminaries: {\em Succ}, {\em Gen} and {\em Kill} Sets}

\begin{frame}[fragile]
	\tableofcontents[currentsection]
\end{frame}

\begin{frame}[fragile,t]
   \frametitle{Prioritized Rules for Liveness}

\bigskip

\begin{description}

    \item[1)] If a variable, {\sc var}, is used, i.e., its value, in an instruction, I,
                \emph{then {\sc var} is {\em live} at the entry of I.}\bigskip

    \item[2)] If {\sc var} is assigned a value in instruction I (and \emph{1)} does not
                apply) \emp{then {\sc var} is {\em dead} at the entry of I.}\bigskip

    \item[3)] If {\sc var} is {\em live} at the end of instruction I \emph{then it is live at the
                entry of I} (unless \emph{2)} applies).\bigskip

    \item[4)] \emph{A {\sc var} is {\em live} at the end of instruction I} $\Leftrightarrow$ 
                \emp{{\sc var} is {\em live} at the
                entry of any instructions that may be executed
                immediately after I}, i.e., \blue{immediate successors of I.}

\end{description}

\end{frame}


\begin{frame}[fragile,t]
   \frametitle{Liveness-Analysis Concepts}

\bigskip

\emp{We number program instructions from $1$ to $n$.}\smallskip

For each instruction we define the following sets:\smallskip

\begin{description}

    \item[\mbox{$succ[i]$:}] The instructions (numbers) that can possibly be
                executed immediately after instruction (numbered) $i$.\smallskip


    \item[\mbox{$gen[i]$:}] The set of variables whose values are read by instruct $i$.\smallskip


    \item[\mbox{$kill[i]$:}]The set of variables that are overwritten by instruction $i$.\smallskip


    \item[\mbox{$in[i]$:}] The set of variables that are live at the entry of instrct $i$.\smallskip


    \item[\mbox{$out[i]$:}] The set of variables that are live at the end of instruct $i$.

\end{description}

\bigskip

\emp{In the end, what we need is $out[i]$ for all instructions.}

\end{frame}



\begin{frame}[fragile,t]
   \frametitle{Immediate Successors}

\bigskip
\bigskip

\begin{itemize}

    \item \emp{$succ[i] = \{i+1\}$} unless instruction $i$ is a {\sc goto}, an
            {\sc if-then-else}, or the last instruction of the program.\bigskip


    \item \emp{succ[i] = \{j\}}, if instruction $i$ is: {\tt GOTO $l$}\\
            {\tt~~~~~~~~~~}and instruction $j$ is: {\tt LABEL $l$}.\bigskip


    \item \emp{succ[i] = \{j, k\}}, if instruction $i$ is {\tt IF c THEN $l_1$ ELSE $l_2$},
            instruction $j$ is {\tt LABEL $l_1$} , and instruction $k$ is {\tt LABEL $l_2$}.\bigskip


    \item If $n$ denotes the last instruction of the program, and $n$ is not a
                {\sc goto} or an {\sc if-then-else} instruction, then \emp{$succ[n] = \emptyset$}.

\end{itemize}

\end{frame}



\begin{frame}
\frametitle{Rules for Constructing $gen$ and $kill$ Sets}

\renewcommand{\arraystretch}{0.9}
\begin{center}
\begin{tabular}{|l|c|c|} \hline
Instruction $i$ & $gen[i]$ & $kill[i]$ \\\hline\hline
${\tt LABEL}~l$ & $\emptyset$ & $\emptyset$ \\\hline
$x := y$ & $\{y\}$ & $\{x\}$ \\\hline
$x := k$ & $\emptyset$ & $\{x\}$ \\\hline
$x := {\bf unop}~y$ & $\{y\}$ & $\{x\}$ \\\hline
$x := {\bf unop}~k$ & $\emptyset$ & $\{x\}$ \\\hline
$x := y~{\bf binop}~z$ & $\{y,z\}$ & $\{x\}$ \\\hline
$x := y~{\bf binop}~k$ & $\{y\}$ & $\{x\}$ \\\hline
$x := M[y]$ & $\{y\}$ & $\{x\}$ \\\hline
$x := M[k]$ & $\emptyset$ & $\{x\}$ \\\hline
$M[x]$ := y& $\{x,y\}$ & $\emptyset$ \\\hline
$M[k]$ := y& $\{y\}$ & $\emptyset$ \\\hline
${\tt GOTO}~l$ & $\emptyset$ & $\emptyset$ \\\hline
${\tt IF}~x~{\bf relop}~y~{\tt THEN}~l_t~{\tt ELSE}~l_f$ & $\{x,y\}$ & $\emptyset$ \\\hline
$x := {\tt CALL}~f(args)$ & $args$ & $\{x\}$ \\\hline
\end{tabular}
\end{center}

\end{frame}

%%%%%%%%%%%%%%%%%%%%%%%%%%%%%%%%%%%%%%%%%%%%%%%%%%%%%%%%%%%%%%%%%%%%%%
%%%%%%%%%%%%%%%%%%%%%%%%%%%%%%%%%%%%%%%%%%%%%%%%%%%%%%%%%%%%%%%%%%%%%%
\section{Liveness Analysis: Equations, Fix-Point Iteration and Interference}

\begin{frame}[fragile]
	\tableofcontents[currentsection]
\end{frame}


\begin{frame}
\frametitle{Data-Flow Equations for Liveness Analysis}

\alert{Let us model the Liveness Rules via Equations!} (Go Back 4 Slides!)

\bigskip


\begin{eqnarray}
in[i] & = & gen[i] \cup (out[i] \setminus kill[i]) \label{in-eq}\\
out[i] & = & \bigcup_{j \in succ[i]} in[j] \label{out-eq}
\end{eqnarray}

\bigskip

\emp{Exception:} If $succ[i] = \emptyset$, then $out[i]$ is the set of variables that
appear in the function's result.

\bigskip

\emp{The (recursive) equations are solved by iterating to a fix point:}\\
$in[i]$ and $out[i]$ are initialized to $\emptyset$, and iterate until no changes occur.

\bigskip

\alert{Why does it converge?}

\bigskip

For fast(er) convergence: compute $out[i]$ before $in[i]$ and $in[i+1]$
before $out[i]$, respectively (i.e., \emp{backward flow analysis}).

\end{frame}



\begin{frame}
\frametitle{Imperative-Fibonacci Example}

\renewcommand{\arraystretch}{0.9}

\[\begin{array}{rl}\\
\scriptstyle{1:} & a := 0 \\
\scriptstyle{2:} & b := 1 \\
\scriptstyle{3:} & z := 0 \\
\scriptstyle{4:} & {\tt LABEL}~loop \\
\scriptstyle{5:} & {\tt IF}~n=z~{\tt THEN}~end~{\tt ELSE}~body \\
\scriptstyle{6:} & {\tt LABEL}~body \\
\scriptstyle{7:} & t := a+b \\
\scriptstyle{8:} & a := b \\
\scriptstyle{9:} & b := t \\
\scriptstyle{10:} & n := n-1 \\
\scriptstyle{11:} & z := 0 \\
\scriptstyle{12:} & {\tt GOTO}~loop \\
\scriptstyle{13:} & {\tt LABEL}~end
\end{array}\begin{array}{|r|c|c|c|} \hline
i & succ[i] & gen[i] & kill[i] \\\hline\hline
1 & 2 &  & a \\\hline
2 & 3 &  & b \\\hline
3 & 4 &  & z \\\hline
4 & 5 &  &  \\\hline
5 & 6,13 & n,z & \\\hline
6 & 7 &  &  \\\hline
7 & 8 & a,b & t \\\hline
8 & 9 & b & a \\\hline
9 & 10 & t & b \\\hline
10 & 11 & n & n \\\hline
11 & 12 &  & z \\\hline
12 & 4 &  &  \\\hline
13 &  &  &  \\\hline
\end{array}\]

Computes {\tt a = fib(n)}. \alert{What would it mean if $in[1] \not= \{n\}$?}

\end{frame}



\begin{frame}
\frametitle{Fix-Point Iteration for the Fibonacci Example}

\renewcommand{\arraystretch}{0.9}

\newlength{\savesep}
\setlength{\savesep}{\arraycolsep}
\setlength{\arraycolsep}{0.65ex}

\[\hspace{-4ex}\begin{array}{|r||c|c||c|c||c|c||c|c|} \hline
    & \multicolumn{2}{c||}{\mbox{Initial}}
    & \multicolumn{2}{c||}{\mbox{Iteration 1}}
    & \multicolumn{2}{c||}{\mbox{Iteration 2}}
    & \multicolumn{2}{c|}{\mbox{Iteration 3}} \\

$i$ & out[i] & in[i] &
      out[i] & in[i] &
      out[i] & in[i] &
      out[i] & in[i] \\\hline\hline

1   &        &       &
      n,a    & n     &
      n,a    & n     &
      n,a    & n     \\\hline

2   &        &       &
      n,a,b  & n,a   &
      n,a,b  & n,a   &
      n,a,b  & n,a   \\\hline

3   &        &       &
     n,z,a,b & n,a,b &
     n,z,a,b & n,a,b &
     n,z,a,b & n,a,b \\\hline

4   &        &       &
     n,z,a,b & n,z,a,b &
     n,z,a,b & n,z,a,b &
     n,z,a,b & n,z,a,b \\\hline

5   &        &       &
     a,b,n   & n,z,a,b &
     a,b,n   & n,z,a,b &
     a,b,n   & n,z,a,b \\\hline

6   &        &       &
     a,b,n   & a,b,n &
     a,b,n   & a,b,n &
     a,b,n   & a,b,n \\\hline

7   &        &       &
     b,t,n   & a,b,n &
     b,t,n   & a,b,n &
     b,t,n   & a,b,n \\\hline

8   &        &       &
     t,n     & b,t,n &
     t,n,a   & b,t,n &
     t,n,a   & b,t,n \\\hline

9   &        &       &
     n       & t,n   &
     n,a,b   & t,n,a &
     n,a,b   & t,n,a \\\hline

10   &        &       &
              & n     &
     n,a,b    & n,a,b &
     n,a,b    & n,a,b \\\hline

11   &        &       &
              &       &
     n,z,a,b  & n,a,b &
     n,z,a,b  & n,a,b \\\hline

12   &        &       &
              &       &
     n,z,a,b  & n,z,a,b &
     n,z,a,b  & n,z,a,b \\\hline

13   &        &       &
       a      & a     &
       a      & a     &
       a      & a     \\\hline
\end{array}\]

\setlength{\arraycolsep}{\savesep}

Usually less than $5$ iterations.

\end{frame}



%%%%%%%%%%%%%%%%%%%%%%%%%%%%%%%%%%%%%%%%%%%%%%%%%%%%%%%%%%%%%%%%%%%%%%
%%%%%%%%%%%%%%%%%%%%%%%%%%%%%%%%%%%%%%%%%%%%%%%%%%%%%%%%%%%%%%%%%%%%%%
\section{Register-Allocation via Coloring: Interference Graph \& Intuitive Alg}

\begin{frame}[fragile]
	\tableofcontents[currentsection]
\end{frame}

\begin{frame}
\frametitle{Interference}

\emph{Definition:} Variable $x$ \emp{interferes} with variable $y$, 
if there is an instruction numbered $i$ such that:

\smallskip

\begin{enumerate}
\item Instruction $i$ is not of the form $x := y$ \emp{and}
\item $x \in kill[i]$ \emp{and}
\item $y \in out[i]$  \emp{and}
\item $x \neq y$
\end{enumerate}

\bigskip

\emph{Two variables can share the same register iff they do not interfere
with each other!}

\end{frame}




\begin{frame}
\frametitle{Interference for the Fibonacci Example}

\renewcommand{\arraystretch}{0.9}
\[\begin{array}{|r|c|c|} \hline
\mbox{Instruction} & \mbox{Left-hand side} & \mbox{Interferes with} \\\hline
1 & a & n \\\hline
2 & b & n,a \\\hline
3 & z & n,a,b \\\hline
7 & t & b,n \\\hline
8 & a & t,n \\\hline
9 & b & n,a \\\hline
10 & n & a,b \\\hline
11 & z & n,a,b \\\hline
\end{array}\]

We can draw interference as a graph:

\begin{center}
\setlength{\unitlength}{0.5ex}

\begin{picture}(70,35)(0,7)

\put(20,20){\makebox(0,0){$a$}}

\put(23,18){\line(2,-1){14}}

\put(23,20){\line(1,0){34}}

\put(23,21.5){\line(3,2){24.5}}

\put(22,23){\line(1,2){7}}

\put(40,10){\makebox(0,0){$b$}}

\put(39,13){\line(-1,3){8}}

\put(43,12){\line(2,1){14}}

\put(41,13){\line(1,3){8}}

\put(60,20){\makebox(0,0){$n$}}

\put(58,23){\line(-1,2){7}}

\put(57,21.5){\line(-3,2){24.5}}

\put(30,40){\makebox(0,0){$z$}}

\put(50,40){\makebox(0,0){$t$}}

\end{picture}
\end{center}

\end{frame}




\begin{frame}
\frametitle{Register Allocation By Graph Coloring}

\emp{Two variables connected by an edge in the interference graph cannot
share a register!}

\bigskip


Idea: Associate variables with register numbers such that:

\smallskip

\begin{enumerate}

\item Two variables connected by an edge receive different numbers.

\item Numbers represent the (limited number of) hardware registers.

\end{enumerate}

\bigskip

Equivalent to {\em graph-coloring problem}: color each node with one of $n$
(available) colors, such that any two neighbors are colored differently.

\bigskip

Since \emp{graph coloring is NP complete}, we use a \emph{heuristic method} that
gives good results in most cases.

\bigskip

\emph{{\em Idea:} a node with less-than-$n$ neighbors can always be colored.\\
Eliminate such nodes from the graph and solve recursively!}

\end{frame}



\begin{frame}
\frametitle{Farvning af eksempelgraf med fire farver}

\only<1>{
\begin{center}
\setlength{\unitlength}{0.7ex}

\begin{picture}(70,35)(0,7)
\put(20,20){\makebox(0,0){$a$}}
\put(23,18){\line(2,-1){14}}
\put(23,20){\line(1,0){34}}
\put(23,21.5){\line(3,2){24.5}}
\put(22,23){\line(1,2){7}}
\put(40,10){\makebox(0,0){$b$}}
\put(39,13){\line(-1,3){8}}
\put(43,12){\line(2,1){14}}
\put(41,13){\line(1,3){8}}
\put(60,20){\makebox(0,0){$n$}}
\put(58,23){\line(-1,2){7}}
\put(57,21.5){\line(-3,2){24.5}}
\put(30,40){\makebox(0,0){$z$}}
\put(50,40){\makebox(0,0){$t$}}
\end{picture}
\end{center}

\emp{$z$ and $t$ have only three neighbors so they can wait.}

}
%
\only<2>{
\begin{center}
\setlength{\unitlength}{0.7ex}

\begin{picture}(70,35)(0,7)
\put(20,20){\makebox(0,0){$a$}}
\put(23,18){\line(2,-1){14}}
\put(23,20){\line(1,0){34}}
\put(23,21.5){\line(3,2){24.5}}
\put(22,23){\line(1,2){7}}
\put(40,10){\makebox(0,0){$b$}}
\put(39,13){\line(-1,3){8}}
\put(43,12){\line(2,1){14}}
\put(41,13){\line(1,3){8}}
\put(60,20){\makebox(0,0){$n$}}
\put(58,23){\line(-1,2){7}}
\put(57,21.5){\line(-3,2){24.5}}
\put(30,40){\makebox(0,0){\textcolor{lightgray}{$z$}}}
\put(50,40){\makebox(0,0){\textcolor{lightgray}{$t$}}}
\end{picture}
\end{center}

\emph{The remaining three nodes can now be given different colors!}

}
%
\only<3>{
\begin{center}
\setlength{\unitlength}{0.7ex}

\begin{picture}(70,35)(0,7)
\put(20,20){\makebox(0,0){\textcolor{red}{$a$}}}
\put(23,18){\line(2,-1){14}}
\put(23,20){\line(1,0){34}}
\put(23,21.5){\line(3,2){24.5}}
\put(22,23){\line(1,2){7}}
\put(40,10){\makebox(0,0){\textcolor{green}{$b$}}}
\put(39,13){\line(-1,3){8}}
\put(43,12){\line(2,1){14}}
\put(41,13){\line(1,3){8}}
\put(60,20){\makebox(0,0){\textcolor{blue}{$n$}}}
\put(58,23){\line(-1,2){7}}
\put(57,21.5){\line(-3,2){24.5}}
\put(30,40){\makebox(0,0){\textcolor{lightgray}{$z$}}}
\put(50,40){\makebox(0,0){\textcolor{lightgray}{$t$}}}
\end{picture}
\end{center}

\emph{$z$ and $t$ can now be given a different color!}

}
%
\only<4>{
\begin{center}
\setlength{\unitlength}{0.7ex}

\begin{picture}(70,35)(0,7)
\put(20,20){\makebox(0,0){\textcolor{red}{$a$}}}
\put(23,18){\line(2,-1){14}}
\put(23,20){\line(1,0){34}}
\put(23,21.5){\line(3,2){24.5}}
\put(22,23){\line(1,2){7}}
\put(40,10){\makebox(0,0){\textcolor{green}{$b$}}}
\put(39,13){\line(-1,3){8}}
\put(43,12){\line(2,1){14}}
\put(41,13){\line(1,3){8}}
\put(60,20){\makebox(0,0){\textcolor{blue}{$n$}}}
\put(58,23){\line(-1,2){7}}
\put(57,21.5){\line(-3,2){24.5}}
\put(30,40){\makebox(0,0){\textcolor{yellow}{$z$}}}
\put(50,40){\makebox(0,0){\textcolor{yellow}{$t$}}}
\end{picture}
\end{center}

\emp{But what if we only have three colors (registers) available?}

}

\end{frame}



%%%%%%%%%%%%%%%%%%%%%%%%%%%%%%%%%%%%%%%%%%%%%%%%%%%%%%%%%%%%%%%%%%%%%%
%%%%%%%%%%%%%%%%%%%%%%%%%%%%%%%%%%%%%%%%%%%%%%%%%%%%%%%%%%%%%%%%%%%%%%
\section{Register-Allocation via Coloring: Improved Algorithm with Spilling}

\begin{frame}[fragile]
	\tableofcontents[currentsection]
\end{frame}


\begin{frame}
\frametitle{Improved Algorithm}

\begin{description}

\item[{\bf Initialization}:] Start with an empty stack.\bigskip

\item[{\bf Simplify}:] 1) If there is a node with less than $n$ edges (neighbors):
        (i) place it on the stack together with the list of edges,
        and (ii) remove it and its edges from the graph.\smallskip


  2. \blue{If there is no node with less than $n$ neighbors, pick
        any node and do as above.}\smallskip


  3. Continue until the graph is empty. If so go to {\em select}.\bigskip


\item[{\bf Select}:] 1. Take a node and its neighbor list from the stack.\smallskip

    2. \emph{If possible, color it differently than its neighbor's.}\smallskip

    3. If not possible, select the node for {\em spilling} (fails).\smallskip

    4. Repeat until stack is empty.

\end{description}

\bigskip

The quality of the result depends on \blue{(i) how to chose a node in
{\em simplify}}, and \emph{(ii) how to chose a color in {\em select}}.

\end{frame}





\begin{frame}
\frametitle{Example: Coloring the Graph with Three Colors}


\only<1>{
\begin{center}
\setlength{\unitlength}{0.7ex}

\begin{picture}(80,35)(0,7)
\put(20,20){\makebox(0,0){$a$}}
\put(23,18){\line(2,-1){14}}
\put(23,20){\line(1,0){34}}
\put(23,21.5){\line(3,2){24.5}}
\put(22,23){\line(1,2){7}}
\put(40,10){\makebox(0,0){$b$}}
\put(39,13){\line(-1,3){8}}
\put(43,12){\line(2,1){14}}
\put(41,13){\line(1,3){8}}
\put(60,20){\makebox(0,0){$n$}}
\put(58,23){\line(-1,2){7}}
\put(57,21.5){\line(-3,2){24.5}}
\put(30,40){\makebox(0,0){$z$}}
\put(50,40){\makebox(0,0){$t$}}
\end{picture}

No node has $< 3$ neighbors, hence choose arbitrarily, say $z$.


\renewcommand{\arraystretch}{0.9}
\[\begin{array}{|c|c|c|}\hline
\mbox{Node} & \mbox{Neighbours} & \mbox{Colour} \\\hline
 & &  \\
 & &  \\
 & &  \\
 & &  \\
z & a,b,n & \\\hline
\end{array}\]
\end{center}
}
%
\only<2>{
\begin{center}
\setlength{\unitlength}{0.7ex}

\begin{picture}(80,35)(0,7)
\put(20,20){\makebox(0,0){$a$}}
\put(23,18){\line(2,-1){14}}
\put(23,20){\line(1,0){34}}
\put(23,21.5){\line(3,2){24.5}}
\put(22,23){\line(1,2){7}}
\put(40,10){\makebox(0,0){$b$}}
\put(39,13){\line(-1,3){8}}
\put(43,12){\line(2,1){14}}
\put(41,13){\line(1,3){8}}
\put(60,20){\makebox(0,0){$n$}}
\put(58,23){\line(-1,2){7}}
\put(57,21.5){\line(-3,2){24.5}}
\put(30,40){\makebox(0,0){\textcolor{lightgray}{$z$}}}
\put(50,40){\makebox(0,0){$t$}}
\end{picture}

There are still no nodes with $< 3$ neighbors, hence we chose $a$.


\renewcommand{\arraystretch}{0.9}
\[\begin{array}{|c|c|c|}\hline
\mbox{Node} & \mbox{Neighbours} & \mbox{Colour} \\\hline
 & &  \\
 & &  \\
 & &  \\
a & b,n,t &  \\
z & a,b,n &  \\\hline
\end{array}\]
\end{center}
}
%
\only<3>{
\begin{center}
\setlength{\unitlength}{0.7ex}

\begin{picture}(80,35)(0,7)
\put(20,20){\makebox(0,0){\textcolor{lightgray}{$a$}}}
\put(23,18){\line(2,-1){14}}
\put(23,20){\line(1,0){34}}
\put(23,21.5){\line(3,2){24.5}}
\put(22,23){\line(1,2){7}}
\put(40,10){\makebox(0,0){$b$}}
\put(39,13){\line(-1,3){8}}
\put(43,12){\line(2,1){14}}
\put(41,13){\line(1,3){8}}
\put(60,20){\makebox(0,0){$n$}}
\put(58,23){\line(-1,2){7}}
\put(57,21.5){\line(-3,2){24.5}}
\put(30,40){\makebox(0,0){\textcolor{lightgray}{$z$}}}
\put(50,40){\makebox(0,0){$t$}}
\end{picture}

$b$ has two neighbors, so we choose it.


\renewcommand{\arraystretch}{0.9}
\[\begin{array}{|c|c|c|}\hline
\mbox{Node} & \mbox{Neighbours} & \mbox{Colour} \\\hline
 & &  \\
 & &  \\
b & t,n & \\
a & b,n,t & \\
z & a,b,n & \\\hline
\end{array}\]
\end{center}
}
%
\only<4>{
\begin{center}
\setlength{\unitlength}{0.7ex}

\begin{picture}(80,35)(0,7)
\put(20,20){\makebox(0,0){\textcolor{lightgray}{$a$}}}
\put(23,18){\line(2,-1){14}}
\put(23,20){\line(1,0){34}}
\put(23,21.5){\line(3,2){24.5}}
\put(22,23){\line(1,2){7}}
\put(40,10){\makebox(0,0){\textcolor{lightgray}{$b$}}}
\put(39,13){\line(-1,3){8}}
\put(43,12){\line(2,1){14}}
\put(41,13){\line(1,3){8}}
\put(60,20){\makebox(0,0){$n$}}
\put(58,23){\line(-1,2){7}}
\put(57,21.5){\line(-3,2){24.5}}
\put(30,40){\makebox(0,0){\textcolor{lightgray}{$z$}}}
\put(50,40){\makebox(0,0){$t$}}
\end{picture}

Finally, choose $t$ and $n$.


\renewcommand{\arraystretch}{0.9}
\[\begin{array}{|c|c|c|}\hline
\mbox{Node} & \mbox{Neighbours} & \mbox{Colour} \\\hline
n & &  \\
t & n &  \\
b & t,n &  \\
a & b,n,t &  \\
z & a,b,n &  \\\hline
\end{array}\]
\end{center}
}
%
\only<5>{
\begin{center}
\setlength{\unitlength}{0.7ex}

\begin{picture}(80,35)(0,7)
\put(20,20){\makebox(0,0){\textcolor{lightgray}{$a$}}}
\put(23,18){\line(2,-1){14}}
\put(23,20){\line(1,0){34}}
\put(23,21.5){\line(3,2){24.5}}
\put(22,23){\line(1,2){7}}
\put(40,10){\makebox(0,0){\textcolor{lightgray}{$b$}}}
\put(39,13){\line(-1,3){8}}
\put(43,12){\line(2,1){14}}
\put(41,13){\line(1,3){8}}
\put(60,20){\makebox(0,0){\textcolor{lightgray}{$n$}}}
\put(58,23){\line(-1,2){7}}
\put(57,21.5){\line(-3,2){24.5}}
\put(30,40){\makebox(0,0){\textcolor{lightgray}{$z$}}}
\put(50,40){\makebox(0,0){\textcolor{lightgray}{$t$}}}
\end{picture}

$n$ has no neighbors so we can choose \textcolor{red}{1}.

\renewcommand{\arraystretch}{0.9}
\[\begin{array}{|c|c|c|}\hline
\mbox{Node} & \mbox{Neighbours} & \mbox{Colour} \\\hline
n & & \textcolor{red}{1} \\
t & n &  \\
b & t,n &  \\
a & b,n,t &  \\
z & a,b,n &  \\\hline
\end{array}\]
\end{center}
}
%
\only<6>{
\begin{center}
\setlength{\unitlength}{0.7ex}

\begin{picture}(80,35)(0,7)
\put(20,20){\makebox(0,0){\textcolor{lightgray}{$a$}}}
\put(23,18){\line(2,-1){14}}
\put(23,20){\line(1,0){34}}
\put(23,21.5){\line(3,2){24.5}}
\put(22,23){\line(1,2){7}}
\put(40,10){\makebox(0,0){\textcolor{lightgray}{$b$}}}
\put(39,13){\line(-1,3){8}}
\put(43,12){\line(2,1){14}}
\put(41,13){\line(1,3){8}}
\put(60,20){\makebox(0,0){\textcolor{red}{$n$}}}
\put(58,23){\line(-1,2){7}}
\put(57,21.5){\line(-3,2){24.5}}
\put(30,40){\makebox(0,0){\textcolor{lightgray}{$z$}}}
\put(50,40){\makebox(0,0){\textcolor{lightgray}{$t$}}}
\end{picture}

$t$ only has $n$ as neighbor, so we can color it with \textcolor{blue}{2}.

\renewcommand{\arraystretch}{0.9}
\[\begin{array}{|c|c|c|}\hline
\mbox{Node} & \mbox{Neighbours} & \mbox{Colour} \\\hline
n & & \textcolor{red}{1} \\
t & n & \textcolor{blue}{2} \\
b & t,n & \\
a & b,n,t & \\
z & a,b,n &  \\\hline
\end{array}\]
\end{center}
}
%
\only<7>{
\begin{center}
\setlength{\unitlength}{0.7ex}

\begin{picture}(80,35)(0,7)
\put(20,20){\makebox(0,0){\textcolor{lightgray}{$a$}}}
\put(23,18){\line(2,-1){14}}
\put(23,20){\line(1,0){34}}
\put(23,21.5){\line(3,2){24.5}}
\put(22,23){\line(1,2){7}}
\put(40,10){\makebox(0,0){\textcolor{lightgray}{$b$}}}
\put(39,13){\line(-1,3){8}}
\put(43,12){\line(2,1){14}}
\put(41,13){\line(1,3){8}}
\put(60,20){\makebox(0,0){\textcolor{red}{$n$}}}
\put(58,23){\line(-1,2){7}}
\put(57,21.5){\line(-3,2){24.5}}
\put(30,40){\makebox(0,0){\textcolor{lightgray}{$z$}}}
\put(50,40){\makebox(0,0){\textcolor{blue}{$t$}}}
\end{picture}

$b$ has $t$ and $n$ as neighbors, hence we can color it with  \textcolor{green}{3}.

\renewcommand{\arraystretch}{0.9}
\[\begin{array}{|c|c|c|}\hline
\mbox{Node} & \mbox{Neighbours} & \mbox{Colour} \\\hline
n & & \textcolor{red}{1} \\
t & n & \textcolor{blue}{2} \\
b & t,n & \textcolor{green}{3} \\
a & b,n,t &  \\
z & a,b,n & \\\hline
\end{array}\]
\end{center}
}
%
\only<8>{
\begin{center}
\setlength{\unitlength}{0.7ex}

\begin{picture}(80,35)(0,7)
\put(20,20){\makebox(0,0){\textcolor{lightgray}{$a$}}}
\put(23,18){\line(2,-1){14}}
\put(23,20){\line(1,0){34}}
\put(23,21.5){\line(3,2){24.5}}
\put(22,23){\line(1,2){7}}
\put(40,10){\makebox(0,0){\textcolor{green}{$b$}}}
\put(39,13){\line(-1,3){8}}
\put(43,12){\line(2,1){14}}
\put(41,13){\line(1,3){8}}
\put(60,20){\makebox(0,0){\textcolor{red}{$n$}}}
\put(58,23){\line(-1,2){7}}
\put(57,21.5){\line(-3,2){24.5}}
\put(30,40){\makebox(0,0){\textcolor{lightgray}{$z$}}}
\put(50,40){\makebox(0,0){\textcolor{blue}{$t$}}}
\end{picture}

$a$ has three differently-colored neighbors, so it is marked as {\em spill}.

\renewcommand{\arraystretch}{0.9}
\[\begin{array}{|c|c|c|}\hline
\mbox{Node} & \mbox{Neighbours} & \mbox{Colour} \\\hline
n & & \textcolor{red}{1} \\
t & n & \textcolor{blue}{2} \\
b & t,n & \textcolor{green}{3} \\
a & b,n,t & spill \\
z & a,b,n &  \\\hline
\end{array}\]
\end{center}
}
%
\only<9>{
\begin{center}
\setlength{\unitlength}{0.7ex}

\begin{picture}(80,35)(0,7)
\put(20,20){\makebox(0,0){\textcolor{lightgray}{$a$}}}
\put(23,18){\line(2,-1){14}}
\put(23,20){\line(1,0){34}}
\put(23,21.5){\line(3,2){24.5}}
\put(22,23){\line(1,2){7}}
\put(40,10){\makebox(0,0){\textcolor{green}{$b$}}}
\put(39,13){\line(-1,3){8}}
\put(43,12){\line(2,1){14}}
\put(41,13){\line(1,3){8}}
\put(60,20){\makebox(0,0){\textcolor{red}{$n$}}}
\put(58,23){\line(-1,2){7}}
\put(57,21.5){\line(-3,2){24.5}}
\put(30,40){\makebox(0,0){\textcolor{lightgray}{$z$}}}
\put(50,40){\makebox(0,0){\textcolor{blue}{$t$}}}
\end{picture}

$z$ has colors \textcolor{red}{1} and \textcolor{green}{3} as neighbors, 
hence we can color it with \textcolor{blue}{2}.

\renewcommand{\arraystretch}{0.9}
\[\begin{array}{|c|c|c|}\hline
\mbox{Node} & \mbox{Neighbours} & \mbox{Colour} \\\hline
n & & \textcolor{red}{1} \\
t & n & \textcolor{blue}{2} \\
b & t,n & \textcolor{green}{3} \\
a & b,n,t & spill \\
z & a,b,n & \textcolor{blue}{2} \\\hline
\end{array}\]
\end{center}
}
%
\only<10>{
\begin{center}
\setlength{\unitlength}{0.7ex}

\begin{picture}(80,35)(0,7)
\put(20,20){\makebox(0,0){\textcolor{lightgray}{$a$}}}
\put(23,18){\line(2,-1){14}}
\put(23,20){\line(1,0){34}}
\put(23,21.5){\line(3,2){24.5}}
\put(22,23){\line(1,2){7}}
\put(40,10){\makebox(0,0){\textcolor{green}{$b$}}}
\put(39,13){\line(-1,3){8}}
\put(43,12){\line(2,1){14}}
\put(41,13){\line(1,3){8}}
\put(60,20){\makebox(0,0){\textcolor{red}{$n$}}}
\put(58,23){\line(-1,2){7}}
\put(57,21.5){\line(-3,2){24.5}}
\put(30,40){\makebox(0,0){\textcolor{blue}{$z$}}}
\put(50,40){\makebox(0,0){\textcolor{blue}{$t$}}}
\end{picture}

We are now finished, but we need to {\em spill} $a$.

\renewcommand{\arraystretch}{0.9}
\[\begin{array}{|c|c|c|}\hline
\mbox{Node} & \mbox{Neighbours} & \mbox{Colour} \\\hline
n & & \textcolor{red}{1} \\
t & n & \textcolor{blue}{2} \\
b & t,n & \textcolor{green}{3} \\
a & b,n,t & spill \\
z & a,b,n & \textcolor{blue}{2} \\\hline
\end{array}\]
\end{center}
}
%

\end{frame}




\begin{frame}
\frametitle{Spilling}

\emph{\em Spilling} means that some variables will reside in memory (except for
brief periods). \emp{For each spilled variable:}\smallskip


\begin{itemize}

\item[1)] Select a memory address $addr_x$ , where the value of $x$ will reside.\smallskip

\item[2)] If instruction $i$ uses $x$, then rename it locally to $x_i$.\smallskip

\item[3)] Before an instruction $i$, which reads $x_i$, insert $x_i~:=~M[addr_x]$.\smallskip

\item[4)] After an instruction $i$, which updates $x_i$, insert  $M[addr_x]~:=~x_i$.\smallskip

\item[5)] If $x$ is alive at the beginning of the function/program, insert
            $M[addr_x]~:=~x$ before the first instruction of the function.\smallskip

\item[6)] If $x$ is live at the end of the program/function, insert
            $x~:=~M[addr_x]$ after the last instruction of the function.

\end{itemize}

\bigskip

\emp{Finally, perform liveness analysis and register allocation again.}

\end{frame}





\begin{frame}
\frametitle{Spilling Example}

\renewcommand{\arraystretch}{0.9}
\[\begin{array}{rl}
\scriptstyle{1:} & a_1 := 0 \\
                 & M[address_a] := a_1 \\
\scriptstyle{2:} & b := 1 \\
\scriptstyle{3:} & z := 0 \\
\scriptstyle{4:} & {\tt LABEL}~loop \\
\scriptstyle{5:} & {\tt IF}~n=z~{\tt THEN}~end~{\tt ELSE}~body \\
\scriptstyle{6:} & {\tt LABEL}~body \\
                 & a_7 := M[address_a] \\
\scriptstyle{7:} & t := a_7+b \\
\scriptstyle{8:} & a_8 := b \\
                 & M[address_a] := a_8 \\
\scriptstyle{9:} & b := t \\
\scriptstyle{10:} & n := n-1 \\
\scriptstyle{11:} & z := 0 \\
\scriptstyle{12:} & {\tt GOTO}~loop \\
\scriptstyle{13:} & {\tt LABEL}~end \\
                  & a := M[address_a] \\
\end{array}\]


\end{frame}



\begin{frame}
\frametitle{After Spilling, Coloring Succeeds!}

\begin{center}
\setlength{\unitlength}{0.7ex}

\begin{picture}(70,40)(0,7)
\put(20,30){\makebox(0,0){\textcolor{green}{$a_8$}}}
\put(23,28.5){\line(4,-1){34}}
\put(23,31.5){\line(3,1){24.5}}
\put(20,20){\makebox(0,0){\textcolor{blue}{$a_1$}}}
\put(23,20){\line(1,0){34}}
\put(20,10){\makebox(0,0){\textcolor{blue}{$a_7$}}}
\put(23,10){\line(1,0){14}}
\put(23,11.5){\line(4,1){34}}
\put(40,10){\makebox(0,0){\textcolor{green}{$b$}}}
\put(39,13){\line(-1,3){8}}
\put(43,12){\line(2,1){14}}
\put(41,13){\line(1,3){8}}
\put(60,20){\makebox(0,0){\textcolor{red}{$n$}}}
\put(58,23){\line(-1,2){7}}
\put(57,21.5){\line(-3,2){24.5}}
\put(20,40){\makebox(0,0){\textcolor{red}{$a$}}}
\put(30,40){\makebox(0,0){\textcolor{blue}{$z$}}}
\put(50,40){\makebox(0,0){\textcolor{blue}{$t$}}}
\end{picture}
\end{center}

\end{frame}



\begin{frame}
\frametitle{Heuristics}

\begin{description}
\item[For {\bf Simplify}:] when choosing a node with $\geq n$ neighbors:\smallskip
\begin{itemize}

\item Chose the node with fewest neighbors, which is
        more likely to be colorable, or\smallskip

\item Chose a node with many neighbors, each of them
        having close to $n$ neighbors, i.e., spilling this node
        would allow the coloring of its neighbors.

\end{itemize}

\bigskip

\item[For {\bf Select}:] when choosing a color:\smallskip
\begin{itemize}

\item Chose colors that have already been used.\smallskip

\item If instructions such as  $x:=y$ exist, color $x$ and $y$
        with the same color, i.e., eliminate this instruction.

\end{itemize}

\end{description}

\end{frame}

%%%%%%%%%%%%%%%%%%%%%%%%%%%%%%%%%%%%%%%%%%%%%%%%%%%%%%%%%%%%%%%%%%%%%%
%%%%%%%%%%%%%%%%%%%%%%%%%%%%%%%%%%%%%%%%%%%%%%%%%%%%%%%%%%%%%%%%%%%%%%


\end{document}
