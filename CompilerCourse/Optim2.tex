\documentclass{beamer}

%% \documentclass[handout]{beamer}
%% % use this with the [handout] option to create handouts for the audience
%% \usepackage{pgfpages}
%% \pgfpagesuselayout{2 on 1}[a4paper,border shrink=5mm]

\mode<presentation>
{
  \usetheme{Diku}
% set this to your preferences:
  \setbeamercovered{invisible}
%  \setbeamercovered{transparent}
}

\usepackage{graphicx}
\usepackage{epic}

\usepackage{amsmath}
\usepackage{amssymb}
\usepackage{amsthm}

\newcommand{\basetop}[1]{\vtop{\vskip-1ex\hbox{#1}}}
\newcommand{\source}[1]{\let\thefootnote\relax\footnotetext{\scriptsize\textcolor{kugray1}{Source: #1}}}

% for coloured code citation in text:
\usepackage{fancyvrb}

%%%%%%%%%%%%%%%%%%%%%%%%%%%%%%%%%
%%%%%    code sections   %%%%%%%%
%%%%%%%%%%%%%%%%%%%%%%%%%%%%%%%%%

% code highlighting commands in own block
\DefineVerbatimEnvironment{code}{Verbatim}{fontsize=\scriptsize}
\DefineVerbatimEnvironment{icode}{Verbatim}{fontsize=\scriptsize}

% Fancy code with color commands:
\DefineVerbatimEnvironment{colorcode}%
        {Verbatim}{fontsize=\scriptsize,commandchars=\\\{\}}

%%%%%%%%%%%%%%%%%%%%%%%%%%%%%%%%%%
%%%%%    some coloring    %%%%%%%%

\definecolor{Red}{RGB}{220,50,10}
\definecolor{Blue}{RGB}{0,51,102}
\definecolor{Yellow}{RGB}{102,51,0}
\definecolor{Orange}{RGB}{178,36,36}
\definecolor{Grey}{RGB}{180,180,180}
\definecolor{Green}{RGB}{20,120,20}
\definecolor{Purple}{RGB}{160,50,100}
\newcommand{\red}[1]{\textcolor{Red}{{#1}}}
\newcommand{\blue}[1]{\textcolor{Blue}{{#1}}}
\newcommand{\yellow}[1]{\textcolor{Yellow}{{#1}}}
\newcommand{\orange}[1]{\textcolor{Orange}{{#1}}}
\newcommand{\grey}[1]{\textcolor{Grey}{{#1}}}
\newcommand{\green}[1]{\textcolor{Green}{{#1}}}
\newcommand{\purple}[1]{\textcolor{Purple}{{#1}}}




% use "DIKU green" from our color theme for \emph
\renewcommand{\emph}[1]{\textcolor{structure}{#1}}
% use some not-too-bright red for an \emp command
\definecolor{DikuRed}{RGB}{130,50,32}
\newcommand{\emp}[1]{\textcolor{DikuRed}{ #1}}
\definecolor{CosGreen}{RGB}{10,100,70}
\newcommand{\emphh}[1]{\textcolor{CosGreen}{ #1}}
\definecolor{CosBlue}{RGB}{55,111,122}
\newcommand{\emphb}[1]{\textcolor{CosBlue}{ #1}}
\definecolor{CosRed}{RGB}{253,1,1}
\newcommand{\empr}[1]{\textcolor{CosRed}{ #1}}

\newcommand{\mymath}[1]{$ #1 $}
\newcommand{\myindx}[1]{_{#1}}
\newcommand{\myindu}[1]{^{#1}}

\newcommand{\Fasto}{\textsc{Fasto}\xspace}



%%%%%%%%%%%%%%%%%%%%

\title[Optimizations]{Optimizations}

\author[C.~Oancea]{Cosmin E. Oancea\\{\tt cosmin.oancea@diku.dk}}

\institute{Department of Computer Science (DIKU)\\University of Copenhagen}


\date[December 2012]{December 2012 Compiler Lecture Notes}




\begin{document}

\titleslide

\input{Struct_Interm/StructTACoptim}

\begin{frame}[fragile]
	\tableofcontents
\end{frame}

%%%%%%%%%%%%%%%%%%%%%%%%%%%%%%%%%%%%%%%%%%%%%%%%%%%%%%%%%%%%%%%%%
%%%%%%%%%%%%%%%%%%%%%%%%%%%%%%%%%%%%%%%%%%%%%%%%%%%%%%%%%%%%%%%%%
%%%%%%%%%%%%%%%%%%%%%%%%%%%%%%%%%%%%%%%%%%%%%%%%%%%%%%%%%%%%%%%%%
\section{Data-Flow Analysis}



%%%%%%%%%%%%%%%%%%%%%%%%%%%%%%%%%%%%%%%%%%%%%%%%%%%%%%%%%%%%%%%%%
\subsection{Common-Subexpression Elimination}


\begin{frame}[fragile,t]
    \frametitle{Data-Flow Analysis}

Global analysis is used to derive information that can drive
optimizations. Example: {\em Liveness analysis.}


\bigskip

Information can flow forwards or backwards through the program.\\
Typical Structure:\smallskip

\begin{itemize}
    \item Successor/predecessor on basic blocks ($succ[B_i]$ or $pred[B_i]$).\smallskip
    \item Define $gen[i]$ and $kill[i]$ sets.\smallskip
    \item Define equations for $in[i]$ and $out[i]$.\smallskip
    \item Initialize $in[i]$ and $out[i]$.\smallskip
    \item Iterate to a fix point.\smallskip
    \item Use $in[i]$ or $out[i]$ for optimizations.
\end{itemize}

\end{frame}



\begin{frame}
\frametitle{Common-Subexpression elimination}

Goal: remove redundant computations.

\smallskip

Example: statement {\tt a[i] := a[i]+1} translates to:

\smallskip

\vspace{1ex}
\texttt{
\begin{tabular}{l}
t1 := 4*i\\
t2 := a+t1\\
t3 := M[t2]\\
t4 := t3+1\\
t5 := 4*i\\
t6 := a+t5\\
M[t6] := t4
\end{tabular}
}

\bigskip

Potential for optimization:\smallskip

\begin{itemize}
\item Term {\tt 4*i} is computed twice.\smallskip
\item {\tt a+t1} and {\tt a+t5} are equal, since {\tt t1}={\tt t5}. 
\end{itemize}

\end{frame}





\begin{frame}
\frametitle{Available-Assignments Analysis}

\begin{center}
\scalebox{0.75}{\makebox{
\begin{tabular}{|l|c|c|} \hline
Instruction $i$ & $gen[i]$ & $kill[i]$ \\\hline\hline
${\tt LABEL}~l$ & $\emptyset$ & $\emptyset$ \\\hline
$x := y$ & $\emptyset$ & $assg(x)$\\\hline
$x := k$ & $\{x := k\}$ & $assg(x)$ \\\hline
$x := {\bf unop}~y$~~~~ where $x\neq y$ & $\{x := {\bf unop}~y\}$
  & $assg(x)$ \\\hline
$x := {\bf unop}~x$ & $\emptyset$
  & $assg(x)$ \\\hline
$x := {\bf unop}~k$ &  $\{x := {\bf unop}~k\}$
  & $assg(x)$ \\\hline
$x := y~{\bf binop}~z$~~~~where $x\neq y$ and $x\neq z$
  & $\{x := y~{\bf binop}~z\}$
  & $assg(x)$ \\\hline
$x := y~{\bf binop}~z$~~~~ where $x=y$ or $x=z$ 
  &  $\emptyset$
  & $assg(x)$ \\\hline
$x := y~{\bf binop}~k$~~~~ where $x\neq y$
  & $\{x := y~{\bf binop}~k\}$
  & $assg(x)$ \\\hline
$x := x~{\bf binop}~k$
  & $\emptyset$
  & $assg(x)$ \\\hline
$x := M[y]$ ~~~~ where $x\neq y$ & $\{x := M[y]\}$ & $assg(x)$ \\\hline
$x := M[x]$ & $\emptyset$ & $assg(x)$ \\\hline
$x := M[k]$ & $\{x := M[k]\}$  & $assg(x)$ \\\hline
$M[x]$ := y&  $\emptyset$  & $loads$ \\\hline
$M[k]$ := y&  $\emptyset$  & $loads$ \\\hline
${\tt GOTO}~l$ & $\emptyset$ & $\emptyset$ \\\hline
${\tt IF}~x~{\bf relop}~y~{\tt THEN}~l_t~{\tt ELSE}~l_f$
 & $\emptyset$  & $\emptyset$ \\\hline
$x := {\tt CALL}~f(args)$ & $\emptyset$  & $assg(x)$  \\\hline
\end{tabular}
}}
\end{center}

$assg(x)$: Assignements that use $x$ on the left or right-hand sides,\\
$loads$: Statements of form $y := M[\cdot]$.

\end{frame}

\begin{frame}
\frametitle{Example for Available Assignments}

\scalebox{0.8}{\makebox{$\begin{array}{rl}\\
\scriptstyle{1:} & i := 0 \\
\scriptstyle{2:} & a := n*3 \\
\scriptstyle{3:} & {\tt IF}~i<a~{\tt THEN}~loop~{\tt ELSE}~end \\
\scriptstyle{4:} & {\tt LABEL}~loop \\
\scriptstyle{5:} & b := i*4 \\
\scriptstyle{6:} & c := p+b \\
\scriptstyle{7:} & d := M[c] \\
\scriptstyle{8:} & e := d*2 \\
\scriptstyle{9:} & f := i*4 \\
\scriptstyle{10:} & g := p+f \\
\scriptstyle{11:} & M[g] := e \\
\scriptstyle{12:} & i := i+1 \\
\scriptstyle{13:} & a := n*3 \\
\scriptstyle{14:} & {\tt IF}~i<a~{\tt THEN}~loop~{\tt ELSE}~end \\
\scriptstyle{15:} & {\tt LABEL}~end
\end{array}~~~~~~~~~
\begin{array}{|r|c|c|c|} \hline
i & pred[i] & gen[i] & kill[i] \\\hline\hline
1 &   & 1 & 1,5,9,12 \\\hline
2 & 1 & 2 & 2 \\\hline
3 & 2 &  &  \\\hline
4 & 3,14 &  &  \\\hline
5 & 4 & 5 & 5,6 \\\hline
6 & 5 & 6 & 6,7 \\\hline
7 & 6 & 7 & 7,8 \\\hline
8 & 7 & 8 & 8 \\\hline
9 & 8 & 9 & 9,10 \\\hline
10 & 9 & 10 & 10 \\\hline
11 & 10 &  & 7 \\\hline
12 & 11 &  & 1,5,9,12 \\\hline
13 & 12 & 2 & 2 \\\hline
14 & 13 &  &  \\\hline
15 & 3,14 &  &  \\\hline
\end{array}
$}}

\bigskip


Note: Assignment 2 and assignment 13 both represented by 2.

\end{frame}



\begin{frame}
\frametitle{Fix-Point Iteration}

\begin{eqnarray}
out[i] & = & gen[i] \cup (in[i] \setminus kill[i])\label{aa-out}\\
in[i] & = & \bigcap_{j \in pred[i]} out[j]\label{aa-in}
\end{eqnarray}

Initialized to the set of all assignments, except for $in[1] = \emptyset$.

\vspace{1ex}
\scalebox{0.55}{\makebox{$\begin{array}{|r||c|c||c|c||c|c|} \hline
    & \multicolumn{2}{c||}{\mbox{Initialisation}}
    & \multicolumn{2}{c||}{\mbox{Iteration 1}}
    & \multicolumn{2}{c|}{\mbox{Iteration 2}} \\

$i$ & in[i] & out[i] &
      in[i] & out[i] &
      in[i] & out[i] \\\hline\hline

1   &        & 1,2,5,6,7,8,9,10 &
             & 1      &
             & 1     \\\hline

2   & 1,2,5,6,7,8,9,10 & 1,2,5,6,7,8,9,10 &
      1                & 1,2      &
      1                & 1,2     \\\hline

3   & 1,2,5,6,7,8,9,10 & 1,2,5,6,7,8,9,10 &
      1,2              & 1,2      &
      1,2              & 1,2     \\\hline

4   & 1,2,5,6,7,8,9,10 & 1,2,5,6,7,8,9,10 &
      1,2              & 1,2      &
      2                & 2       \\\hline

5   & 1,2,5,6,7,8,9,10 & 1,2,5,6,7,8,9,10 &
      1,2              & 1,2,5      &
      2                & 2,5       \\\hline

6   & 1,2,5,6,7,8,9,10 & 1,2,5,6,7,8,9,10 &
      1,2,5            & 1,2,5,6    &
      2,5              & 2,5,6     \\\hline

7   & 1,2,5,6,7,8,9,10 & 1,2,5,6,7,8,9,10 &
      1,2,5,6          & 1,2,5,6,7  &
      2,5,6            & 2,5,6,7   \\\hline

8   & 1,2,5,6,7,8,9,10 & 1,2,5,6,7,8,9,10 &
      1,2,5,6,7        & 1,2,5,6,7,8 &
      2,5,6,7          & 2,5,6,7,8  \\\hline

9   & 1,2,5,6,7,8,9,10 & 1,2,5,6,7,8,9,10 &
      1,2,5,6,7,8      & 1,2,5,6,7,8,9 &
      2,5,6,7,8        & 2,5,6,7,8,9  \\\hline

10  & 1,2,5,6,7,8,9,10 & 1,2,5,6,7,8,9,10 &
      1,2,5,6,7,8,9    & 1,2,5,6,7,8,9,10 &
      2,5,6,7,8,9      & 2,5,6,7,8,9,10  \\\hline

11  & 1,2,5,6,7,8,9,10 & 1,2,5,6,7,8,9,10 &
      1,2,5,6,7,8,9,10    & 1,2,5,6,8,9,10      &
      2,5,6,7,8,9,10      & 2,5,6,8,9,10 \\\hline

12  & 1,2,5,6,7,8,9,10 & 1,2,5,6,7,8,9,10 &
      1,2,5,6,8,9,10      & 2,6,8,10    &
      2,5,6,8,9,10        & 2,6,8,10   \\\hline

13  & 1,2,5,6,7,8,9,10 & 1,2,5,6,7,8,9,10 &
      2,6,8,10         & 2,6,8,10    &
      2,6,8,10         & 2,6,8,10  \\\hline

14  & 1,2,5,6,7,8,9,10 & 1,2,5,6,7,8,9,10 &
      2,6,8,10         & 2,6,8,10    &
      2,6,8,10         & 2,6,8,10  \\\hline

15  & 1,2,5,6,7,8,9,10 & 1,2,5,6,7,8,9,10 &
      2                & 2          &
      2                & 2         \\\hline

\end{array}
$}}

\vspace{1ex}
Iteration 3 $=$ iteration 2.

\end{frame}

\begin{frame}
\frametitle{Used in Common-Subexpression Elimination}

The computation ~$\scriptstyle{5:}~ \blue{b := i*4}$~ is available at
~$\scriptstyle{9:}~ \red{f := i*4}$.

\vspace{1ex}

\scalebox{0.8}{\makebox{$
\begin{array}{rl}
\scriptstyle{1:} & i := 0 \\
\scriptstyle{2:} & \blue{a := n*3} \\
\scriptstyle{3:} & {\tt IF}~i<a~{\tt THEN}~loop~{\tt ELSE}~end \\
\scriptstyle{4:} & {\tt LABEL}~loop \\
\scriptstyle{5:} & \blue{b := i*4} \\
\scriptstyle{6:} & c := p+b \\
\scriptstyle{7:} & d := M[c] \\
\scriptstyle{8:} & e := d*2 \\
\scriptstyle{9:} & \mbox{\textcolor{red}{$f := \only<1>{i*4}\only<2->{b}$}} \\
\scriptstyle{10:} & g := p+f~~~~\only<3>{\textcolor{red}{\leftarrow~~\mbox{will not be eliminated}}}\\
\scriptstyle{11:} & M[g] := e \\
\scriptstyle{12:} & i := i+1 \\
\scriptstyle{13:} & \textcolor{red}{a := \only<1>{n*3}\only<2->{a}} \\
\scriptstyle{14:} & {\tt IF}~i<a~{\tt THEN}~loop~{\tt ELSE}~end \\
\scriptstyle{15:} & {\tt LABEL}~end
\end{array}
$}}

\end{frame}



%%%%%%%%%%%%%%%%%%%%%%%%%%%%%%%%%%%%%%%%%%%%%%%%%%%%%%%%%%%%%%%%%
\subsection{Jump-to-Jump Elimination}
\begin{frame}[fragile]
    \tableofcontents[currentsubsection]
\end{frame}



\begin{frame}
\frametitle{Jump-to-Jump Elimination}

Avoid successive jumps: {\tt [... GOTO $l_1$,..., LABEL $l_1$, GOTO $l_2$...]}.

{\footnotesize
\begin{center}
\begin{tabular}{|l|l|l|}\hline
  instruction & $gen$ & $kill$ \\\hline
  {\tt LABEL}~$l$ & $\{l\}$ & $\emptyset$ \\\hline
  {\tt GOTO}~$l$ & $\emptyset$ & $\emptyset$ \\\hline
  {\tt IF $c$ THEN $l_1$ ELSE $l_2$} & $\emptyset$ & $\emptyset$ \\\hline
  any other & $\emptyset$ & the set of all labels \\\hline
\end{tabular}
\end{center}

\begin{eqnarray}
in[i] & = & \left\{ {\begin{array}{ll}
                      gen[i]\setminus kill[i] & \mbox{if $out[i]$ is empty} \\
                      out[i]\setminus kill[i] & \mbox{if $out[i]$ is non-empty}
                     \end{array}}
            \right.
 \label{jump-in}\\
out[i] & = & \bigcap_{j \in succ[i]} in[j] \label{jump-out}
\end{eqnarray}
}

\vspace{1ex}
A jump ~$\scriptstyle{i:}~{\tt GOTO}~l$~ can be replaced with ~$\scriptstyle{i:}~{\tt GOTO}~l'$, if $l' \in in[i]$.

\end{frame}



%%%%%%%%%%%%%%%%%%%%%%%%%%%%%%%%%%%%%%%%%%%%%%%%%%%%%%%%%%%%%%%%%
\subsection{Index-Checking Elimination}

\begin{frame}[fragile]
    \tableofcontents[currentsubsection]
\end{frame}


\begin{frame}
\frametitle{Index-Check Elimination}

Checks if {\tt i} is within the array-size bounds when used in {\tt a[i]}.\smallskip

Idea: use {\tt IF-THEN-ELSE} to check bounds and analyze whether the
condition reduces statically to {\em true} or {\em false}.\smallskip

Example: If {\tt a}'s lowest/highest index is $0/10$, translate ~{\tt
for~i:=0~to~9~do~a[i]:=0;} ~~~~to:
\vspace{1ex}

\scalebox{0.7}{\makebox{$\begin{array}{rl}
\scriptstyle{1:} & i := 0 \\
\scriptstyle{2:} & {\tt LABEL}~for1\\
\scriptstyle{3:} & {\tt IF}~i\leq 9~{\tt THEN}~for2~{\tt ELSE}~for3 \\
\scriptstyle{4:} & {\tt LABEL}~for2 \\
\scriptstyle{5:} & \textcolor{red}{{\tt IF}~i<0~{\tt THEN}~error~{\tt ELSE}~ok1} \\
\scriptstyle{6:} & {\tt LABEL}~ok1 \\
\scriptstyle{7:} & \textcolor{red}{{\tt IF}~i>10~{\tt THEN}~error~{\tt ELSE}~ok2} \\
\scriptstyle{8:} & {\tt LABEL}~ok2 \\
\scriptstyle{9:} & t := i*4 \\
\scriptstyle{10:} & t := a+t \\
\scriptstyle{11:} & M[t] := 0 \\
\scriptstyle{12:} & i := i+1 \\
\scriptstyle{13:} & {\tt GOTO}~for1 \\
\scriptstyle{14:} & {\tt LABEL}~for3
\end{array}$}}

\end{frame}

\begin{frame}
\frametitle{Inequalities}

Collect inequalities of the form $p \leq q$ and $p<q$, where $p$ and $q$ are
either variables or constants.\smallskip

In order to ensure a finite number of inequalities, use an universe $Q$ of
inequalities derived from program's condition(al)s. For example:\smallskip

\begin{itemize}

\item $when(x < 10) = \{x < 10\}$, $whennot(x<10)=\{10 \leq x\}$.\smallskip

\item $when(x = y)$ = $\{x \leq y,\, y \leq x\}$, $whennot(x=y)=\emptyset$.

\end{itemize}

\bigskip

Our example program provides the following universe:

\[Q = \{i\leq 9,\,9<i,\,i<0,\,0\leq i,10<i,\,i\leq 10\}\]

Fixpoint-iteration computes $in[i] = {}$ the set of inequalities (from $Q$) 
that are true (hold) at the beginning of instruction $i$.

\end{frame}



\begin{frame}
\frametitle{Equations for Inequalities}

\scalebox{0.8}{\makebox{$
in[i] = \left\{
\begin{array}{l}
\bigcap_{j \in pred[i]} in[j] \\
~~~~~~~~~~ \mbox{if $pred[i]$ has more than one element} \\
in[pred[i]] \cup when(c) \\
~~~~~~~~~~ \mbox{if $pred[i]$ is {\tt IF\,$c$\,THEN\,$i$\,ELSE\,$j$}} \\
in[pred[i]] \cup whennot(c) \\
~~~~~~~~~~ \mbox{if $pred[i]$ is {\tt IF\,$c$\,THEN\,$j$\,ELSE\,$i$}} \\
(in[pred[i]] \setminus conds(Q,x)) \cup equal(Q, x, p) \\
~~~~~~~~~~ \mbox{if $pred[i]$ is of the form $x:=p$} \\
in[pred[i]] \setminus upper(Q,x) \\
~~~~~~~~~~ \mbox{if $pred[i]$ is of the form $x:=x+k$ where $k\geq 0$} \\
in[pred[i]] \setminus lower(Q,x) \\
~~~~~~~~~~ \mbox{if $pred[i]$ is of the form $x:=x-k$ where $k\geq 0$} \\
in[pred[i]] \setminus conds(Q,x) \\
~~~~~~~~~~ \mbox{if $pred[i]$ is of a form $x:=e$ not covered above} \\
in[pred[i]]\\
~~~~~~~~~~ otherwise \\
\end{array}
\right.
$}}

\vspace{0.5ex}
{\footnotesize
$conds(Q,x)$: inequalities in $x$\newline
$upper(Q,x)$: inequalities of form $x<p$ or $x\leq p$\newline
$lower(Q,x)$: inequalities of form $p<x$ or $p\leq x$\newline
$equal(Q,x,p)$: inequalities from $Q$, which are consequences of $x=p$
}
\end{frame}

\begin{frame}
\frametitle{Limitations of Data-Flow Analysis}

\bigskip

Can never be exact:\smallskip

\begin{itemize}

\item many analysis problems are undecidable, i.e., one cannot solve
        them accurately.

\item Tradeoff between efficient computation and precision.

\item Use conservative approximation: optimize only when you are sure
        the assumptions hold.

\end{itemize}

\end{frame}





%%%%%%%%%%%%%%%%%%%%%%%%%%%%%%%%%%%%%%%%%%%%%%%%%%%%%%%%%%%%%%%%%
%%%%%%%%%%%%%%%%%%%%%%%%%%%%%%%%%%%%%%%%%%%%%%%%%%%%%%%%%%%%%%%%%
\section{Loop Optimizations}

\begin{frame}[fragile]
    \tableofcontents[currentsection]
\end{frame}

%%%%%%%%%%%%%%%%%%%%%%%%%%%%%%%%%%%%%%%%%%%%%%%%%%%%%%%%%%%%%%%%%
\subsection{Hoisting Loop-Invariant Computation}

\begin{frame}
\frametitle{Hoisting Loop-Invariant Computation}

A term is loop invariant if it is computed inside a loop but has the
same value at each iteration.\smallskip

Solution: Unroll the loop once and do common-subexpression
elimination.\smallskip

The loop-invariant term is now computed in the unrolled part and
reused in the subsequent loop.   \alert{Example:}

\definecolor{light}{gray}{0.99}
\vspace{1ex}
\begin{tabular}{lcl}
\scalebox{0.8}{\begin{minipage}{4cm}
{\tt
while ($c$) \{\newline
\textcolor{light}{xx}$body$\newline
\}
}
\end{minipage}}
&$\longrightarrow$&
\scalebox{0.8}{\begin{minipage}{4cm}
{\tt
if ($c$) \{\newline
\textcolor{light}{xx}$body$;\newline
\textcolor{light}{xx}while ($c$) \{\newline
\textcolor{light}{xxxx}$body$\newline
\textcolor{light}{xx}\}\newline
\}
}
\end{minipage}}\\
\textcolor{light}{xx}&&\\
\scalebox{0.8}{\begin{minipage}{4cm}
{\tt
do\newline
\textcolor{light}{xx}$body$\newline
while ($c$)
}
\end{minipage}}

&$\longrightarrow$&
\scalebox{0.8}{\begin{minipage}{4cm}
{\tt
$body$;\newline
while ($c$) \{\newline
\textcolor{light}{xx}$body$\newline
\}
}
\end{minipage}}
\end{tabular}


\bigskip
Disadvantage: Code size.

\end{frame}


%%%%%%%%%%%%%%%%%%%%%%%%%%%%%%%%%%%%%%%%%%%%%%%%%%%%%%%%%%%%%%%%%
\subsection{Prefetching}
%\begin{frame}[fragile]
%\tableofcontents[currentsubsection]
%\end{frame}

\begin{frame}
\frametitle{Memory Prefetching}

\definecolor{light}{gray}{0.99}

\scalebox{0.8}{\begin{minipage}{10cm}
{\tt
sum = 0;\newline
for (i=0, i<100000; i++) \{\newline
\textcolor{light}{xx}sum += a[i];\newline
\}
}
\end{minipage}
}

\bigskip

Problem: the array does not fit in the cache so we constantly wait (for {\tt IO}).
%
Some architectures offer a {\em prefetch} instruction that downloads from
memory into cache and is safe under addressing errors:\smallskip

\vspace{1ex}
\scalebox{0.8}{\begin{minipage}{10cm}
{\tt
sum = 0;\newline
for (i=0, i<100000; i++) \{\newline
\textcolor{light}{xx}\emp{if (i\&3==0)} \blue{prefecth(a[i+32]);}\newline
\textcolor{light}{xx}sum += a[i];\newline
\}
}
\end{minipage}
}

\bigskip
NB!: We assumed that the cache-line size is 4 words.\\\smallskip
We can deploy loop body to avoid testing.

\end{frame}




%%%%%%%%%%%%%%%%%%%%%%%%%%%%%%%%%%%%%%%%%%%%%%%%%%%%%%%%%%%%%%%%%
%%%%%%%%%%%%%%%%%%%%%%%%%%%%%%%%%%%%%%%%%%%%%%%%%%%%%%%%%%%%%%%%%
\section{Function Calls}

\begin{frame}[fragile]
    \tableofcontents[currentsection]
\end{frame}


%%%%%%%%%%%%%%%%%%%%%%%%%%%%%%%%%%%%%%%%%%%%%%%%%%%%%%%%%%%%%%%%%
\subsection{Inlining}


\begin{frame}
\frametitle{Inlining}

A function call:
\scalebox{0.8}{\makebox{
{\tt
$x$ = f($exp_1$,\ldots,$exp_n$);
}}}

\definecolor{light}{gray}{0.99}
\vspace{1.5ex}
where {\tt f} is declared as:

\vspace{1.5ex}
\scalebox{0.8}{\begin{minipage}{10cm}
{\tt
$type_0$ f($type_1$ $x_1$,\ldots,$type_n$ $x_n$)\newline
\{\newline
\textcolor{light}{xx}$body$\\
\textcolor{light}{xx}return($exp$);\newline
\}
}
\end{minipage}}

\bigskip
can be replaced with:

\bigskip
\scalebox{0.8}{\begin{minipage}{10cm}
{\tt
\{\newline
\textcolor{light}{xx}$type_1$ $x_1$ = $exp_1$;\newline
\textcolor{light}{xx}\ldots\newline
\textcolor{light}{xx}$type_n$ $x_n$ = $exp_n$;\newline
\textcolor{light}{xx}$body$\newline
\textcolor{light}{xx}$x$ = $exp$;\newline
\}
}
\end{minipage}
}

\bigskip
Variables need to be renamed whenever necessary, e.g., rename
$x_1 \ldots x_n$ as needed.

\end{frame}


%%%%%%%%%%%%%%%%%%%%%%%%%%%%%%%%%%%%%%%%%%%%%%%%%%%%%%%%%%%%%%%%%
\subsection{Tail-Call Optimization}

%\begin{frame}[fragile]
%   \tableofcontents[currentsubsection]
%\end{frame}


\begin{frame}
\frametitle{Tail-Call Optimization}


A tail call is a call happening just before returning from the (current) function, {\tt f},
e.g., {\tt return(g(x,y));}\smallskip

We use the following observations:\smallskip

\begin{itemize}
\item None of {\tt f}'s variables are {\em live} after the call.\smallskip

\item If {\tt f}'s epilogue is empty (except for the return jump), then 
{\tt g} can return directly to {\tt f}'s return address.\smallskip

\item Since {\tt f}'s activation record contains nothing useful at this point,
we can reuse the space for {\tt g}'s activation record.

\end{itemize}

\bigskip

Hence the program is more efficient in both runtime and memory space.\smallskip

Tail-call optimization is very important for functional languages as it
makes tail recursion as efficient as loops.

\end{frame}

\begin{frame}
\frametitle{Tail-Call Example}

{\tt return(g(x,y));}~~ via stack-based {\em caller saves}.

\vspace{1ex}

\scalebox{0.8}{\makebox{$\begin{array}{ll}
\only<1>{M[FP+4*m+4] := R0}\only<2>{\textcolor{red}{M[FP+4*m+4] := R0}}
 & \only<2>{\longleftarrow~\mbox{Eliminated since no variables are live}} \\
\only<1>{\cdots}\only<2>{\textcolor{red}{\cdots}} \\
\only<1>{M[FP+4*m+4*(k+1)] := Rk}\only<2>{\textcolor{red}{M[FP+4*m+4*(k+1)] := Rk}} \\
\only<1-2>{FP := FP + framesize}\only<3>{\textcolor{red}{FP := FP + framesize}}
 & \only<3>{\longleftarrow~\mbox{Eliminated when we recycle activation records}} \\
M[FP + 4] := {\tt x} \\
M[FP + 4*n] := {\tt y} \\
\only<1-4>{M[FP] := returnaddress}\only<5>{\textcolor{red}{M[FP] := returnaddress}}
 & \only<5>{\longleftarrow~\mbox{Eliminated when we recycle the return address}} \\
\only<1-7>{{\tt GOTO}~{\tt g}} \\
\only<1-4>{{\tt LABEL}~returnaddress}\only<5>{\textcolor{red}{{\tt LABEL}~returnaddress}} \\
\only<1-3>{result := M[FP+4]}\only<4>{\textcolor{red}{result := M[FP+4]}}
 & \only<4>{\longleftarrow~\mbox{Going off against each other (copy propagation)}} \\
\only<1-2>{FP := FP - framesize}\only<3>{\textcolor{red}{FP := FP - framesize}}\\
\only<1>{R0 := M[FP+4*m+4]}\only<2>{\textcolor{red}{R0 := M[FP+4*m+4]}} \\
\only<1>{\cdots}\only<2>{\textcolor{red}{\cdots}} \\
\only<1>{Rk := M[FP+4*m+4*(k+1)]}\only<2>{\textcolor{red}{Rk := M[FP+4*m+4*(k+1)]}}\\
\only<1-3>{M[FP+4] := result}\only<4>{\textcolor{red}{M[FP+4] := result}} \\
\only<1-5>{{\tt GOTO}~M[FP]}\only<6>{\textcolor{red}{{\tt GOTO}~M[FP]}}
 & \only<6>{\longleftarrow~\mbox{Dead code}}
\end{array}$}}

\end{frame}



%%%%%%%%%%%%%%%%%%%%%%%%%%%%%%%%%%%%%%%%%%%%%%%%%%%%%%%%%%%%%%%%%
%%%%%%%%%%%%%%%%%%%%%%%%%%%%%%%%%%%%%%%%%%%%%%%%%%%%%%%%%%%%%%%%%
\section{Specialization}

\begin{frame}[fragile]
    \tableofcontents[currentsection]
\end{frame}


\begin{frame}
\frametitle{Specialisering}

Ideea: Specialize version of functions for constant-value parameters.\\
Example:


\definecolor{light}{gray}{0.99}
\vspace{1.5ex}

\begin{tabular}{ll}
\scalebox{0.8}{\begin{minipage}{7cm}
{\tt
double power(double x, int n)\newline
\{ \newline
\textcolor{light}{xx}double p=1.0; \newline
\textcolor{light}{xx}while (n>0)\newline
\textcolor{light}{xxxx}if (n\%2 == 0) \{\newline
\textcolor{light}{xxxxxx}x = x*x;\newline
\textcolor{light}{xxxxxx}n = n/2;\newline
\textcolor{light}{xxxx}\} else \{\newline
\textcolor{light}{xxxxxx}p = p*x;\newline
\textcolor{light}{xxxxxx}n = n-1;\newline
\textcolor{light}{xxxx}\}\newline
\textcolor{light}{xx}return(p);\newline
\}}
\end{minipage}}
&
\scalebox{0.8}{\begin{minipage}{7cm}
{\tt
\only<1>{\textcolor{light}{xx}}\only<2->{double power5(double x)}\newline
\only<2->{\{}\newline
\only<2->{\textcolor{light}{xx}double p=1.0;}\newline
\only<3->{\textcolor{light}{xxxxxx}p = p*x;}\newline
\only<4->{\textcolor{light}{xxxxxx}x = x*x;}\newline
\only<5->{\textcolor{light}{xxxxxx}x = x*x;}\newline
\only<6->{\textcolor{light}{xxxxxx}p = p*x;}\newline
\only<7->{\textcolor{light}{xx}return(p);}\newline
\only<7->{\}
}}
\end{minipage}}
\end{tabular}

\bigskip
Specialized version of power for parameter {\tt n=5}.

\only<8>{Specialization is the implementation method for C{\tt{}++} templates.}
\end{frame}


%%%%%%%%%%%%%%%%%%%%%%%%%%%%%%%%%%%%%%%%%%%%%%%%%%%%%%%%%%%%%%%%%
%%%%%%%%%%%%%%%%%%%%%%%%%%%%%%%%%%%%%%%%%%%%%%%%%%%%%%%%%%%%%%%%%


\end{document}
